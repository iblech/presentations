\documentclass[a4paper,english,16pt]{scrartcl}

\usepackage[utf8]{inputenc}
\usepackage[ngerman]{babel}

\usepackage{amsmath,amsthm,amsfonts,amssymb}
\usepackage{framed}
\usepackage{setspace}
\usepackage{enumitem}

\newcommand{\E}{\text{E}} % Euklidischer Raum
\newcommand{\R}{\mathbb{R}} % Reelle Zahlen
\newcommand{\conv}{\mathsf{conv}} % Konvexe Hülle
\newcommand{\dist}{\mathsf{dist}} % Distanz

\theoremstyle{definition}
\newtheorem*{theorem}{Theorem}
\newtheorem*{lemma}{Lemma}

\renewcommand{\familydefault}{\sfdefault} % Sans-Serif-Schriftart
\pagestyle{empty} % Keine Seitenzahlen

\begin{document}
\onehalfspacing

\begin{framed}
  \begin{theorem}[Kirchberger', 8.2]\mbox{}\\
    Seien $P$ und $Q$ nichtleere, kompakte Teilmengen von $\E^n$. \\
    Dann sind $P$ und $Q$ genau dann durch eine Sphäre streng trennbar,
    wenn für jede Menge $T \subset \E^n$ mit höchstens $n + 3$ Punkten die Mengen $P \cap T$ und $Q \cap T$ durch eine Sphäre
    streng trennbar sind.
  \end{theorem}
\end{framed}

\vspace{96pt}

\begin{framed}
  \begin{theorem}[9.5]\mbox{}\\
    Seien $P$ und $Q$ nichtleere, kompakte Teilmengen von $\E^n$.\\
    Angenommen, für $1 \leq k \leq n$ kann jede Teilmenge von $Q$ mit höchstens $k$ Punkten streng von $P$ mit einer Hyperebene getrennt werden. Dann gibt es zu jedem $k$-Zylinder $Z_1 = (\conv P) + F_1$ einen $(k{-}1)$-Zylinder $Z_2 = (\conv P) + F_2$ mit $Z_2 \subset Z_1$ und $Z_2 \cap Q = \emptyset$.
  \end{theorem}
\end{framed}

\begin{framed}
  \begin{lemma}[9.4]\mbox{}\\
    Sei $S = S_1(0)$ die Einheitssphäre um den Nullpunkt im $\E^n$ und $F = \{ A_i \mid i \in I \}$ eine Familie von kompakten, stark konvexen Teilmengen von $S$. Angenommen, je $n$ (oder weniger) Elemente von $F$ haben einen Punkt gemeinsam. Dann gibt es ein Paar von antipodalen Punkten $\{ p, {-}p \}$, sodass $\{ p, {-}p \} \cap A_i \not= \emptyset$ für alle $i \in I$.
  \end{lemma}
\end{framed}

\newpage

\begin{framed}
  \begin{theorem}[10.2]\mbox{}\\
    Seien $P$ und $Q$ nichtleere, kompakte Teilmengen von $\E^n$ ($n \geq 2$).\\
    Dann sind für eine Basis $\beta$ von $\E^n$ äquivalent:
    \begin{enumerate}[label=(\alph*)]
      \item Es gibt eine $\beta$-Box $B$, sodass $P \subset B$ und $Q \cap B = \emptyset$.
      \item Für jede Teilmenge $S \subset P {\cup} Q$ mit maximal $n{+}1$ Punkten gibt es eine $\beta$-Box $B_S$, sodass $(P \cap S) \subset B_S$ und $(Q \cap S) \cap B_S = \emptyset$.
      \item Für jede Teilmenge $T \subset P$ mit maximal $n$ Punkten ist die minimale $\beta$-Box $B_T$, die $T$ enthält, disjunkt von $Q$, also \\ $B_T \cap Q = \emptyset$.
    \end{enumerate}
  \end{theorem}
\end{framed}

\end{document}