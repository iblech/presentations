\documentclass{beamer}

\usepackage[utf8]{inputenc}
\usepackage[ngerman]{babel}

\usepackage{tikz}
\usetikzlibrary{3d,calc}
\usepackage{fix-cm}
\usepackage{tikz-3dplot}
\usetikzlibrary{decorations.pathreplacing}

% http://stackoverflow.com/questions/1670463/latex-change-margins-of-only-a-few-pages
\newenvironment{changemargin}[2]{%
  \begin{list}{}{%
    \setlength{\topsep}{0pt}%
    \setlength{\leftmargin}{#1}%
    \setlength{\rightmargin}{#2}%
    \setlength{\listparindent}{\parindent}%
    \setlength{\itemindent}{\parindent}%
    \setlength{\parsep}{\parskip}%
  }%
  \item[]}{
  \end{list}
}

\usetheme{Frankfurt}
%\beamertemplatenavigationsymbolsempty
%\setbeamertemplate{mini frames}{}
\useoutertheme[subsection = false]{miniframes}
\usepackage{color,graphicx,overpic}
\definecolor{DarkBlue}{rgb}{0.098,0.098,0.349} % RGB(25, 25, 89)
\setbeamercolor{section in head/foot}{fg=white, bg=DarkBlue}
\definecolor{LighterBlue}{rgb}{0.15,0.2,0.8}
\setbeamercolor{frametitle}{bg=LighterBlue, fg=white}
\setbeamercolor{title}{fg=white, bg=LighterBlue}
\setbeamercolor{structure}{fg=LighterBlue, bg=white}

%\renewcommand\familydefault{\sfdefault} %comment to see the difference
%\DeclareMathAlphabet      {\mathup}{OT1}{\familydefault}{m}{n}

\definecolor{DarkGreen}{rgb}{0.15,0.7,0.15}

\newcommand{\E}{\text{E}} % Euklidischer Raum
\newcommand{\R}{\mathbb{R}} % Reelle Zahlen
\newcommand{\fa}[1]{\forall \, {#1} \,:\,}
\newcommand{\ex}[1]{\exists \, {#1} \,:\,}
\newcommand{\conv}{\mathsf{conv}} % Konvexe Hülle
\newcommand{\dist}{\mathsf{dist}} % Distanz
\usepackage{mathtools}
\DeclarePairedDelimiter\norm{\lVert}{\rVert}%

\let\mySum\sum
\DeclareMathOperator*{\textsum}{{\textstyle \mySum}}
\renewcommand{\sum}{\textsum\limits}

% http://tex.stackexchange.com/questions/48381/line-through-two-points-with-offset-in-tikz
\tikzset{
    add/.style args={#1 and #2}{
        to path={%
  ($(\tikztostart)!-#1!(\tikztotarget)$)--($(\tikztotarget)!-#2!(\tikztostart)$)%
  \tikztonodes},add/.default={.2 and .2}}
}

% http://tex.stackexchange.com/questions/6135/how-to-make-beamer-overlays-with-tikz-node
\tikzset{onslide/.code args={<#1>#2}{
  \only<#1>{\pgfkeysalso{#2}} % \pgfkeysalso doesn't change the path
}}
\tikzset{temporal/.code args={<#1>#2#3#4}{
  \temporal<#1>{\pgfkeysalso{#2}}{\pgfkeysalso{#3}}{\pgfkeysalso{#4}} % \pgfkeysalso doesn't change the path
}}

\tikzstyle{point}=[circle,fill,inner sep=1pt]
\tikzstyle{cover}=[LightGray]

\definecolor{LightGray}{rgb}{0.8,0.8,0.8}
\newcommand<>{\cover}[1]{\alt#2{{\color{LightGray} {#1}}}{#1}}

% s/Theorem/Satz ?
% Kirchberger-ähnliche Theoreme?
% Theoreme vom Kirchberger-Typ?
% Varianten des Theorems von Kirchberger
\title{Varianten des Theorems von Kirchberger}
\author{Tim Baumann}
\institute{TopMath-Frühlingsschule in Oberschönenfeld}
\date{4. März 2014}

\begin{document}

\setlength{\abovedisplayskip}{2pt}
\setlength{\belowdisplayskip}{2pt}
\setlength{\abovedisplayshortskip}{2pt}
\setlength{\belowdisplayshortskip}{2pt}

\begin{frame}[plain]
  \titlepage
\end{frame}

\begin{frame}[plain]
  \begin{theorem}[Kirchberger]
    Seien $P$ und $Q$ nichtleere, kompakte Teilmengen von $\E^n$. \\
    Dann sind $P$ und $Q$ genau dann durch eine \alert<2>{Hyperebene} streng trennbar, wenn für jede Menge $T \subset \E^n$ mit höchstens $n + 2$ Punkten die Mengen $P \cap T$ und $Q \cap T$ durch eine \alert<2>{Hyperebene} streng trennbar sind.
  \end{theorem}
\end{frame}

\section{Trennung durch Sphären}
\begin{frame}
  \frametitle{Übersicht}
  \tableofcontents[currentsection]
\end{frame}

\begin{frame}
  \begin{definition}
    Sei $p \in \E^n$ und $\alpha > 0$. Dann heißt
    \[ S_\alpha(p) := \left\{ x \in \E^n \mid \norm{x - p} = \alpha \right\} \]
    \emph{Sphäre} mit Radius $\alpha$ um den Punkt $p$.
  \end{definition}
  \begin{definition}<2->
    Seien $A$ und $B$ Teilmengen von $\E^n$.\\
    Die Sphäre $S_\alpha(p)$ \emph{trennt} $A$ und $B$ \emph{streng}, wenn gilt:
    \begin{minipage}{0.39 \linewidth}
      \begin{uncoverenv}<3->
        \vspace{10pt}
        \begin{center}
          \colorbox{white}{
            \begin{tikzpicture}
              \node (A) at (0, 0) {A};
              \draw (0, 0) circle (0.4);
              \node (B) at (1, 0) {B};
            \end{tikzpicture}
          }
        \end{center}
      \end{uncoverenv}
      \begin{uncoverenv}<1->
        \vspace{-8pt}
        \[ \fa{a \in A} \norm{p - a} < \alpha \]
        \vspace{-26pt}

        \begin{center}
          und
        \end{center}
        \vspace{-28pt}

        \[ \fa{b \in B} \norm{p - a} > \alpha \]
        \vspace{-10pt}
      \end{uncoverenv}
    \end{minipage}
    \begin{minipage}{0.19 \linewidth}
      \begin{uncoverenv}<4->
        \vspace{8pt}
        \begin{center}
          oder
        \end{center}
      \end{uncoverenv}
    \end{minipage}
    \begin{minipage}{0.39 \linewidth}
      \begin{uncoverenv}<4->
        \vspace{10pt}
        \begin{center}
          \colorbox{white}{
            \begin{tikzpicture}
              \node (A) at (0, 0) {A};
              \draw (1, 0) circle (0.4);
              \node (B) at (1, 0) {B};
            \end{tikzpicture}
          }
        \end{center}
      \end{uncoverenv}
      \begin{uncoverenv}<4->
        \vspace{-8pt}
        \[ \fa{a \in A} \norm{p - a} > \alpha \]
        \vspace{-26pt}

        \begin{center}
          und
        \end{center}
        \vspace{-28pt}

        \[ \fa{b \in B} \norm{p - a} < \alpha \]
        \vspace{-10pt}
      \end{uncoverenv}
    \end{minipage}
  \end{definition}
\end{frame}

\begin{frame}
  \begin{theorem}[Kirchberger\only<2->{', 8.2}]
    Seien $P$ und $Q$ nichtleere, kompakte Teilmengen von $\E^n$. \\
    Dann sind $P$ und $Q$ genau dann durch eine \alt<2->{\alert<2>{Sphäre}}{Hyperebene} streng trennbar,
    wenn für jede Menge $T \subset \E^n$ mit höchstens \alt<3->{\alert<3->{$n + 3$}}{$n + 2$} Punkten die Mengen $P \cap T$ und $Q \cap T$ durch eine \alt<2->{\alert<2>{Sphäre}}{Hyperebene}\\
    streng trennbar sind.
  \end{theorem}
\end{frame}

\begin{frame}
  Folgendes Beispiel im $\E^2$ zeigt, dass die Trennbarkeit von $n + 2 = 4$ Punkten aus $P = \{ p_1, p_2, p_3 \}$ und $Q = \{ q_1, q_2 \}$ mit Sphären nicht ausreicht, um Trennbarkeit von $P$ und $Q$ zu folgern:

  \begin{center}
    \begin{tikzpicture}
      \path[use as bounding box] (-2,2.5) rectangle (6,-1);
      \draw<4> [gray,dashed,add= .5 and .5] (q1) to (q2);

      \node (p1) [point,DarkGreen,onslide=<5-6>{cover},label=180:\cover<5-6>{$p_1$}] at (0, 2) {};
      \node (p2) [point,DarkGreen,label=-90:$p_2$] at (0, -2) {};
      \node (p3) [point,DarkGreen,onslide=<7-8>{cover},label=-45:\cover<7-8>{$p_3$}] at (2, 0) {};
      \draw [LightGray] (p1) -- (p2);
      \draw [LightGray] (p2) -- (p3);
      \draw [LightGray] (p3) -- (p1);
      \node<2-> (q1) [violet,point,onslide=<9-10>{cover},label=225:\cover<9-10>{$q_1$}] at (0, 0) {};
      \node<3-> (q2) [violet,point,onslide=<11-12>{cover},label=45:\cover<11-12>{$q_2$}] at (4, 0) {};

      \draw<6,13> (2, -2) [blue,dashed] circle (2.25);
      \draw<8,13> (2, 0) [blue,dashed] circle (2.25);
      \draw<10,13> (4, 0) [blue,dashed] circle (1);
      \draw<12,13> (0, 0) [blue,dashed] circle (1);
    \end{tikzpicture}
  \end{center}
\end{frame}

\begin{frame}
  \begin{theorem}[Kirchberger', 8.2]
    Seien $P$ und $Q$ nichtleere, kompakte Teilmengen von $\E^n$. \\
    Dann sind $P$ und $Q$ genau dann durch eine Sphäre streng trennbar,
    wenn für jede Menge $T \subset \E^n$ mit höchstens $n + 3$ Punkten die Mengen $P \cap T$ und $Q \cap T$ durch eine Sphäre\\
    streng trennbar sind.
  \end{theorem}
\end{frame}

\begin{changemargin}{-0.5cm}{-0.5cm}

\begin{frame}
  \vspace{-16pt}
  \begin{center}
    \begin{tikzpicture}
      [grid/.style={very thin,LightGray},
      axis/.style={->,gray}]

      \path[use as bounding box] (0,1.5,0) -- (6.5,0,0) -- (0,0,7);

      \node [blue] at (-3,1,0) {Beweis:};

      % Draw a grid in the x-z plane
      \foreach \x in {0,0.5,...,5}{
        \draw[grid] (\x,0,-0.5) -- (\x,0,6.5);
      }
      \foreach \z in {0,0.5,...,6}{
        \draw[grid] (-0.5,0,\z) -- (5.5,0,\z);
      }

      \draw<2->[axis] (0,0,0) -- (0,1,0) node[anchor=east]{$\E$};
      \draw[axis] (0,0,0) -- (6,0,0) node[anchor=south]{$\E^n$};
      \draw[axis] (0,0,0) -- (0,0,6.5) node[anchor=north]{};

      \shade<3->[ball color=blue!20!white,opacity=0.30] (1.5,1,3) circle (1);

      \node<3-> (p) [point,label=225:\only<3>{$p$}] at (1.5, 0, 3) {};
      \node<3-> (n) [point] at (1.5, 2, 3) {};

      % Find stereographic projection of $x$ with Wolfram Alpha:
      % ||t*{1.5,2,3} + (1-t)*{3,0,5} - {1.5,1,3}|| = 1
      \node<4-5> (x) [point,label=225:$x$] at (3, 0, 5) {};
      \node<5-5> (x') [point,label=45:{$x'=\phi(x)$}] at (2.08537, 1.21951, 3.78049) {};
      \draw<5> [dashed] (x) -- (x');
      \draw<5> [dashed,gray] (x') -- (n);

      \node<6-> (P) [color=DarkGreen] at (1.5, 0, 5) {P};
      \node<6-> (Q) [color=violet] at (4, 0, 5) {Q};
      %\node<3-> (P) [label=225:\cover<9-10>{$p$}] at (1.5, 0, 3) {};

      % Find stereographic projection of $P$ with Wolfram Alpha:
      % ||t*{1.5,2,3} + (1-t)*{1.5,0,5} - {1.5,1,3}|| = 1
      \node<7-> (P') [color=DarkGreen,font=\scriptsize] at (1.5, 1, 4) {P'};
      \draw<7> [dashed] (P) -- (P');
      \draw<7> [dashed,gray] (P') -- (n);

      % Find stereographic projection of $Q$ with Wolfram Alpha:
      % ||t*{1.5,2,3} + (1-t)*{4,0,5} - {1.5,1,3}|| = 1
      \node<7-> (Q') [color=violet,font=\scriptsize] at (2.20176, 1.4386, 3.5614) {Q'};
      \draw<7> [dashed] (Q) -- (Q');
      \draw<7> [dashed,gray] (Q') -- (n);

      \begin{scope}[canvas is zx plane at y=0]
        \draw<10-13> (5, 1.5) circle (0.5);
      \end{scope}

      \begin{scope}[canvas is xy plane at z=3.94]
        \draw<11-13> (1.5, 1) circle (0.24);
        \begin{scope}
          \clip (1.5, 1) circle (0.24) (-5,-5) rectangle (5,5);
          \shade<12-13> [ball color=blue, opacity=0.25] (0.75,0.25) rectangle (2.25,1.75);
        \end{scope}
      \end{scope}

      \draw<11-13> [gray] (1, 0, 5.05) -- (n);
      \draw<11-13> [gray] (2, 0, 4.9) -- (n);

      \begin{scope}[canvas is xy plane at z=3.85]
        \draw<14-> (1.51, 1.05) circle (0.5);
        \begin{scope}
          \clip (1.51, 1.05) circle (0.5) (-5,-5) rectangle (5,5);
          \shade<14-> [ball color=orange, opacity=0.25] (0.75,0.25) rectangle (2.25,1.75);
        \end{scope}
      \end{scope}

      \draw<17-> [gray] (0.2, 0, 5.05) -- (n);
      \draw<17-> [gray] (2.55, 0, 4.65) -- (n);

      \begin{scope}[canvas is zx plane at y=0]
        \draw<17-> (5, 1.4) circle (1.2);
      \end{scope}
    \end{tikzpicture}
  \end{center}

  \vspace{-10pt}

  \only<1-7>{
    \begin{enumerate}
      \item<2-> Bette $\E^n$ wie üblich in den $\E^{n+1}$ ein.
      \item<3-> Sei $p \!\in\! \E^n$ und $S \!\subset\! \E^{n+1}$ eine Sphäre, die in $p$ tangential zu $\E^n$ ist.
      \item<4-> Betrachte die stereographische Projektion $\phi : \E^n \to S$.
      \item<6-> Seien ${\color{DarkGreen} P}, {\color{violet} Q} \subset \E^n$ nichtleer und kompakt sodass für jede Menge $T \subset \E^n$ mit höchstens $n + 3$ Punkten die Mengen $P \cap T$ und $Q \cap T$ durch eine Sphäre streng trennbar sind.
      \item<7-> Seien ${\color{DarkGreen} P' = \phi(P)}$ und ${\color{violet} Q' = \phi(Q)}$ die (kompakten) Bilder von ${\color{DarkGreen} P}$ bzw. ${\color{violet} Q}$ unter $\phi$.
    \end{enumerate}
  }

  \only<8-13>{
    \textbf{Behauptung:} ${\color{DarkGreen} P'}$ und ${\color{violet} Q'}$ können durch eine Hyperebene ${\color{orange} H_0} \subset \E^{n+1}$ streng getrennt werden.
    \vspace{-4pt}

    \begin{enumerate}
      \setcounter{enumi}{5}
      \item<9-> Sei $T \subset S \subset \E^{n+1}$ eine Menge mit höchstens $n + 3$ Punkten.
      \item<10-> Nach Voraussetzung werden die Urbilder $\phi^{-1}(T \!\cap\! P') = \phi^{-1}(T) \cap P$ und $\phi^{-1}(T \cap Q') = \phi^{-1}(T) \cap Q$ durch eine Sphäre streng getrennt.
      \item<11-> Die stereogr. Projektion der Sphäre ist ein Kreis auf $S$ (Kreistreue).
      \item<12-> Der Kreis auf $S$ ist der Schnitt von $S$ mit einer Hyperebene $H$.
      \item<13-> Da $H$ dann $T \cap P'$ und $T \cap Q'$ streng trennt, folgt die Behauptung nach dem Satz von Kirchberger.
    \end{enumerate}
    \vspace{-22pt}
  }

  \only<14-17>{
    \begin{enumerate}
      \setcounter{enumi}{10}
      \item<14-> Sei $\alpha \in \E^{n+1}$ und $b \in \R$, sodass $\langle \alpha, p \rangle < b$ für alle $p \in P'$ und $\langle \alpha, q \rangle > b$ für alle $q \in Q'$.
      \item<15-> Da $P'$ und $Q'$ kompakt sind, gibt es $\epsilon > 0$ mit $\langle \alpha, p \rangle \leq b - \epsilon$ für alle $p \in P'$ und $\langle \alpha, q \rangle \geq b + \epsilon$ für alle $q \in Q'$.
      \item<16-> Somit können wir annehmen, dass $H_0$ den Nordpol der Sphäre $S$ nicht schneidet.
      \item<17-> Der Schnitt $H_0 \cap S$ ist ein Kreis und $\phi^{-1}(H_0 \cap S)$ trennt $P$ und $Q$.
      \hfill $\square$
    \end{enumerate}
  }
\end{frame}

\end{changemargin}

\section{Trennung durch Zylinder}

\begin{frame}
  \frametitle{Übersicht}
  \tableofcontents[currentsection]
\end{frame}

\begin{frame}
  \setbeamercolor{background canvas}{bg=violet}

  \begin{definition}
    Sei $A \!\subset\! \E^n$ und $F \!\subset\! \E^n$ ein $k$-dimensionaler Unterraum. Dann heißt
    \[ Z = A + F = \{ a + f \mid a \in A, f \in F \} \]
    von $A$ und $F$ erzeugter \emph{$k$-Zylinder}.
  \end{definition}

  \begin{tikzpicture}[color=green!50!black]
    \path[use as bounding box] (-5.5,-2.75) rectangle (5.5,2.75);

    \only<2->{
      \draw [dashed] (1,0,0) -- (0,0,-1) -- (-1,0,0);
      \draw (1,0,0) -- (0,0,1) -- (-1,0,0);
      \draw (0,1,0) -- (0,0,1) -- (0,-1,0);
      \draw (0,1,0) -- (1,0,0) -- (0,-1,0);
      \draw [dashed] (0,1,0) -- (0,0,-1) -- (0,-1,0);
      \draw (0,1,0) -- (-1,0,0) -- (0,-1,0);
    }

    \only<2-3>{
      \fill[opacity=0.15] (0,1,0) -- (1,0,0) -- (0,-1,0) -- (0,0,1) -- cycle;
      \fill[opacity=0.15] (0,1,0) -- (-1,0,0) -- (0,-1,0) -- (0,0,-1) -- cycle;
      \fill[opacity=0.25] (0,1,0) -- (1,0,0) -- (0,-1,0) -- (0,0,-1) -- cycle;
      \fill[opacity=0.25] (0,1,0) -- (-1,0,0) -- (0,-1,0) -- (0,0,1) -- cycle;
    }

    \node<3> (Z0) [color=DarkGreen] at (0, -1.5, 0) {0-Zylinder};

    \only<4>{
      \node (Z1) [color=DarkGreen] at (0, -1.5, 0) {1-Zylinder};

      \draw (-5,1,1) -- (-5,0,2) -- (-5,-1,1);
      \draw [dashed] (-5,1,1) -- (-5,0,0) -- (-5,-1,1);
      \draw (5,1,-1) -- (5,0,0) -- (5,-1,-1);
      \draw (5,1,-1) -- (5,0,-2) -- (5,-1,-1);
      \draw (-5,1,1) -- (5,1,-1);
      \draw (-5,0,2) -- (5,0,0);
      \draw (-5,-1,1) -- (5,-1,-1);
      \draw [dashed] (-5,0,0) -- (5,0,-2);

      \fill[opacity=0.25] (-5,0,2) -- (-5,-1,1) -- (5,-1,-1) -- (5,0,0) -- cycle;
      \fill[opacity=0.25] (-5,1,1) -- (-5,0,0) -- (5,0,-2) -- (5,1,-1) -- cycle;
      \fill[opacity=0.15] (-5,1,1) -- (-5,0,2) -- (5, 0, 0) -- (5,1,-1) -- cycle;
      \fill[opacity=0.15] (-5,0,0) -- (-5,-1,1) -- (5,-1,-1) -- (5,0,-2) -- cycle;
    }

    \only<5>{
      \node (Z2) [color=DarkGreen] at (0, -1.5, 0) {2-Zylinder};

      \draw (-5,2,0) -- (-5,-1,3) -- (-5,-2,2);
      \draw [dashed] (-5,2,0) -- (-5,1,-1) -- (-5,-2,2);
      \draw (5,2,-2) -- (5,-1,1) -- (5,-2,0);
      \draw (5,2,-2) -- (5,1,-3) -- (5,-2,0);
      \draw (-5,2,0) -- (5,2,-2);
      \draw (-5,-1,3) -- (5,-1,1);
      \draw (-5,-2,2) -- (5,-2,0);
      \draw [dashed] (-5,1,-1) -- (5,1,-3);

      \fill[opacity=0.25] (-5,-1,3) -- (-5,-2,2) -- (5,-2,0) -- (5,-1,1) -- cycle;
      \fill[opacity=0.25] (-5,2,0) -- (-5,1,-1) -- (5,1,-3) -- (5,2,-2) -- cycle;
      \fill[opacity=0.15] (-5,2,0) -- (-5,-1,3) -- (5, -1, 1) -- (5,2,-2) -- cycle;
      \fill[opacity=0.15] (-5,1,-1) -- (-5,-2,2) -- (5,-2,0) -- (5,1,-3) -- cycle;
    }
  \end{tikzpicture}
\end{frame}

\begingroup
\definecolor{LightGreen}{rgb}{0.7216,0.8667,0.7216}
\setbeamercolor{background canvas}{bg=LightGreen}

\begin{frame}
  \begin{definition}
    Sei $A \!\subset\! \E^n$ und $F \!\subset\! \E^n$ ein $k$-dimensionaler Unterraum. Dann heißt
    \[ Z = A + F = \{ a + f \mid a \in A, f \in F \} \]
    von $A$ und $F$ erzeugter \emph{$k$-Zylinder}.
  \end{definition}

  \begin{tikzpicture}[color=green!50!black]
    \path[use as bounding box] (-5.5,-2.75) rectangle (5.5,2.75);

    \draw [dashed] (1,0,0) -- (0,0,-1) -- (-1,0,0);
    \draw (1,0,0) -- (0,0,1) -- (-1,0,0);
    \draw (0,1,0) -- (0,0,1) -- (0,-1,0);
    \draw (0,1,0) -- (1,0,0) -- (0,-1,0);
    \draw [dashed] (0,1,0) -- (0,0,-1) -- (0,-1,0);
    \draw (0,1,0) -- (-1,0,0) -- (0,-1,0);

    \node (Z3) [color=DarkGreen] at (0, -1.5, 0) {3-Zylinder};
  \end{tikzpicture}
\end{frame}
\endgroup


\begin{frame}
  \frametitle{Kirchberger-Theorem für Zylinder?}

  \begin{theorem}[???]
    Seien $P$ und $Q$ nichtleere, kompakte Teilmengen von $\E^n$.\\
    Dann gibt es einen $k$-Zylinder $Z = (\conv P) + F$ mit $Z \cap Q = \emptyset$ genau dann, wenn es für alle Teilmengen $T \subset P \cup Q$ mit höchstens \alert<2>{$f(n,k)$} Punkten einen $k$-Zylinder $Z_T = \conv(T \cap P) + F_T$ mit $Z_T \cap (T \cap Q) = \emptyset$ gibt.
  \end{theorem}
\end{frame}

\begin{frame}
  \begin{center}
    \begin{tikzpicture}
      \path[use as bounding box] (-5,-4) rectangle (5,4);

      \node<1-6> at (0,3.5) {Dann gilt $f(2,1) \geq 9$:};
      \node<7-8> at (0,1.5) {Dann gilt $f(2,1) \geq 15$:};

      \draw [color=DarkGreen] (0,0) circle (0.5);
      \fill<1,5,7> [color=DarkGreen,opacity=0.20] (0,0) circle (0.5);
      \node (P) [color=DarkGreen] at (0,0) {P};

      \node<1-4> [point,violet,label=0:$q_1$] at (0:2.25) {};
      \node<5-6> [point,LightGray,label=0:{\color{LightGray} $q_1$}] at (0:2.25) {};
      \foreach \num/\angle in {2/40,3/80,4/120,5/160,6/200,7/240,8/280,9/320}
        \node<1-6> [point,violet,label=\angle:$q_{\num}$] at (\angle:2) {};

      \node<7> [point,violet,label=0:$q_1$] at (0:3.25) {};
      \node<8> [point,LightGray,label=0:{\color{LightGray} $q_1$}] at (0:3.25) {};
      \foreach \num/\angle in {2/24,3/48,4/72,5/96,6/120,7/144,8/168,9/192,10/216,11/240,12/264,13/288,14/312,15/336}
        \node<7-8> [point,violet,label=\angle:$q_{\num}$] at (\angle:3.25) {};

      \begin{scope}[cm={cos(3),-sin(3),sin(3),cos(3), (0,0)}]
        \fill<2-4,6> [color=DarkGreen,opacity=0.2] (-10,-0.5) -- (10,-0.5) -- (10,0.5) -- (-10, 0.5) -- cycle;
      \end{scope}
    
      \begin{scope}[cm={cos(45),-sin(45),sin(45),cos(45), (0,0)}]
        \fill<3-4> [color=DarkGreen,opacity=0.2] (-10,-0.5) -- (10,-0.5) -- (10,0.5) -- (-10, 0.5) -- cycle;
      \end{scope}
    
      \begin{scope}[cm={cos(110),-sin(110),sin(110),cos(110), (0,0)}]
        \fill<4> [color=DarkGreen,opacity=0.2] (-10,-0.5) -- (10,-0.5) -- (10,0.5) -- (-10, 0.5) -- cycle;
      \end{scope}

      \begin{scope}[cm={cos(1),-sin(1),sin(1),cos(1), (0,0)}]
        \fill<8> [color=DarkGreen,opacity=0.2] (-10,-0.5) -- (10,-0.5) -- (10,0.5) -- (-10, 0.5) -- cycle;
      \end{scope}
    \end{tikzpicture}
  \end{center}
\end{frame}

\begin{frame}
  \frametitle{Kirchberger-Theorem für Zylinder? So nicht!}

  \begin{theorem}[???]
    Seien $P$ und $Q$ nichtleere, kompakte Teilmengen von $\E^n$.\\
    Dann gibt es einen $k$-Zylinder $Z = (\conv P) + F$ mit $Z \cap Q = \emptyset$ genau dann, wenn es für alle Teilmengen $T$ von $P \cup Q$ mit höchstens \alert<1>{$f(n,k)$} Punkten einen $k$-Zylinder $Z_T = \conv(T \cap P) + F_T$ mit $Z_T \cap (T \cap Q) = \emptyset$ gibt.
  \end{theorem}

  \begin{tikzpicture}[red,ultra thick,line cap=round]
    \path[use as bounding box] (-0.5,-0.25) rectangle (7.5,-0.25);

    \draw (0,0) -- (10,4);
    \draw (0.2,0.1) -- (9.8,3.9);
    \draw (0.1,-0.1) -- (9.9,4.1);
    \draw (0,-0.1) -- (9.8,4.1);
    \draw (0,-0.2) -- (9.8,4);
  \end{tikzpicture}
\end{frame}

\begin{frame}
  \begin{theorem}[9.5]
    Seien $P$ und $Q$ nichtleere, kompakte Teilmengen von $\E^n$.\\
    Angenommen, für $1 \leq k \leq n$ kann jede Teilmenge von $Q$ mit höchstens $k$ Punkten streng von $P$ mit einer Hyperebene getrennt werden. Dann gibt es zu jedem $k$-Zylinder $Z_1 = (\conv P) + F_1$ einen $(k{-}1)$-Zylinder $Z_2 = (\conv P) + F_2$ mit $Z_2 \subset Z_1$ und $Z_2 \cap Q = \emptyset$.
  \end{theorem}
\end{frame}

\begin{frame}
  \begin{block}{Beispiel}
    Seien $P, Q \subset \E^3$ kompakt.
    \begin{itemize}
      \item<2-> Wenn jeder Punkt aus $Q$ mit einer Hyperebene streng von $P$ getrennt werden kann, dann liegt $Q$ außerhalb von $\conv P$.
      \item<3-> Wenn je zwei Punkte aus $Q$ mit einer Hyperebene streng von $P$ getrennt werden können, dann gibt es einen $1$-Zylinder, der $P$ beinhaltet und disjunkt von $Q$ ist.
      \item<4-> Wenn je drei Punkte aus $Q$ mit einer Hyperebene streng von $P$ getrennt werden können, dann gibt es zwei parallele Hypereben, sodass $P$ zwichen ihnen und $Q$ außerhalb liegt.
    \end{itemize}
  \end{block}

  \begin{center}
    \begin{tikzpicture}
      \path[use as bounding box] (-5,-1.5) rectangle (5,1.5);

      % 1. Beispiel
      \draw<2-> (-3.5,0) ++ (-0.5,0.5) -- ++ (0,-1) arc (180:360:0.5) -- ++ (0,1) arc (0:180:0.5);
      \node<2-> [color=DarkGreen] at (-3.5,0) {$P$};
      \node<2-> [color=violet] at (-4.3,0.1) {$Q$};
      \node<2-> [color=violet] at (-2.7,-0.3) {$Q$};
      \node<2-> [color=violet] at (-3.6,-1.3) {$Q$};
      \node<2-> [color=violet] at (-3.7,1.3) {$Q$};

      % 2. Beispiel
      \draw<3-> (-0.4,0) ++ (80:1.4) -- ++ (80:-2.8);
      \draw<3-> (0.4,0) ++ (80:1.4) -- ++ (80:-2.8);
      \draw<3-> (-0.4,0) ++ (80:1.4) ++ (0.4,0) ellipse (0.4 and 0.2);
      \draw<3-> (-0.4,0) ++ (80:-1.4) ++ (0.4,0) ellipse (0.4 and 0.2);
      \node<3-> [color=DarkGreen] at (0,0) {$P$};
      \node<3-> [color=violet] at (-0.8,-0.2) {$Q$};
      \node<3-> [color=violet] at (0.8,0.2) {$Q$};

      % 3. Beispiel
      \node<4-> [color=DarkGreen] at (3.5,0) {$P$};
      \draw<4-> (2.7,-1.5) -- ++ (0,2.4) -- ++ (40:0.6) -- ++ (0,-2.4) -- cycle;
      \draw<4-> (4.0,-1.5) -- ++ (0,2.4) -- ++ (40:0.6) -- ++ (0,-2.4) -- cycle;
      \node<4-> [color=violet] at (2.4,0.1) {$Q$};
      \node<4-> [color=violet] at (4.7,-0.2) {$Q$};
    \end{tikzpicture}
  \end{center}
\end{frame}

\begin{frame}
  \begin{definition}
    Eine Teilmenge $K \subset S_{\alpha}(p)$ heißt \emph{stark konvex}, wenn $K$ keine antipodalen (gegenüberliegenden) Punkte enthält und zu jedem Paar von Punkten auch den kleineren Bogen des Großkreises zwischen diesen Punkten enthält.
  \end{definition}
  \begin{center}
    \begin{tikzpicture}
      \path[use as bounding box] (0,-2) rectangle (10,1.5);

      \node<2-> at (1,-1.5) {Beispiel};
      \draw<2-> [dashed,gray] (1,0) circle (1);
      \draw<2-> [blue, ultra thick, line cap=round] (1,0)++(30:1) arc (30:150:1);

      \node<2-> at (5,-1.5) {Gegenbeispiel};
      \draw<2-> [dashed,gray] (5,0) circle (1);
      \draw<2-> [blue, ultra thick, line cap=round] (5,0)++(30:1) arc (30:75:1);
      \draw<2-> [blue, ultra thick, line cap=round] (5,0)++(105:1) arc (105:150:1);

      \node<2-> at (9,-1.5) {Gegenbeispiel};
      \draw<2-> [dashed,gray] (9,0) circle (1);
      \draw<2-> [blue, ultra thick, line cap=round] (9,0)++(-10:1) arc (-10:190:1);
    \end{tikzpicture}
  \end{center}
\end{frame}

\begin{frame}
  \begin{lemma}[9.4]
    Sei $S = S_1(0)$ die Einheitssphäre um den Nullpunkt im $\E^n$ und $F = \{ A_i \mid i \in I \}$ eine Familie von kompakten, stark konvexen Teilmengen von $S$. Angenommen, je $n$ (oder weniger) Elemente von $F$ haben einen Punkt gemeinsam. Dann gibt es ein Paar von antipodalen Punkten $\{ p, {-}p \}$, sodass $\{ p, {-}p \} {\cap} A_i \not= \emptyset$ für alle $i {\in} I$.
  \end{lemma}
\end{frame}

\begin{frame}
  \begin{theorem}<2->[Horn, 6.8]
    Sei $F = \{ A_i \mid i \in I \}$ eine Familie von kompakten, konvexen Teilmengen von $\E^n$ mit mindestens $n$ Elementen. Angenommen, jede Unterfamilie mit $k$ Elementen besitzt einen gemeinsamen Punkt, wobei $1 \leq k \leq n$. Dann gibt es für jeden $(n{-}k)$-dimensionalen Unterraum $F_1$ einen $(n{-}k{+}1)$-dimensionalen Unterraum $F_2$, sodass $F_2 \supset F_1$ und $F_2 \cap A_i \not= \emptyset$ für alle $i \in I$.
  \end{theorem}

  \begin{proof}[Beweis von Lemma 9.4]
    \begin{enumerate}
      \item<1-> Für alle $i \in I$ gilt: Da $A_i \subset S$ kompakt und stark konvex ist, ist $\conv A_i$ kompakt und enthält nicht den Nullpunkt.
      \item<3-> Aus dem Lemma von Horn folgt mit $k {=} n$, $F_1 {=} \{ 0 \}$, dass ein 1-dimensionaler Unterraum $L$ mit $L \cap \conv A_i \not= \emptyset$ existiert.
      \item<4-> Da $A_i$ stark konvex ist, gilt auch $L \cap A_i \not= \emptyset$ für alle $i \in I$.
      \item<5-> Mit $\{ p, {-}p \} \coloneqq L \cap S$ folgt die Aussage.\qedhere
    \end{enumerate}
  \end{proof}
\end{frame}

\begin{frame}
  \small
  \begin{proof}[Beweis von Theorem 9.5]
    \renewcommand{\qedsymbol}{}
    \vspace{-12pt}
    \begin{flalign*}
      \text{Sei} \enspace \delta \coloneqq \inf \{ \dist(\conv T, \conv P) \mid \, &T \text{ ist Teilmenge von $Q$ mit } &\\[-4pt]
      &\text{höchstens $k$ Punkten } \}. &
    \end{flalign*}
    \vspace{-16pt}

    \textbf{Behauptung}: $\delta > 0$\\
    \vspace{-4pt}
    \begin{enumerate}
      \setlength{\itemsep}{0pt}
      \setlength{\parskip}{0pt}
      \item<2-> Sei $R$ die Menge aller $x \in \E^n$, die Konvexkombination von höchstens $k$ Punkten aus $Q$ sind.
      Die Menge $R$ ist kompakt, da sie Bild von
      \begin{align*}
        Q^k \times M^k \to \E^n, \qquad &(q_1, ..., q_k, \lambda_1, ..., \lambda_k) \mapsto \lambda_1 q_1 + ... + \lambda_k q_k,\\
        \text{mit } M^k \coloneqq &\{ (\lambda_1, ..., \lambda_k) \in \left[0,1\right]^k \mid \lambda_1 + ... + \lambda_k = 1 \},
      \end{align*}
      einer stetigen Abbildung mit kompakter Definitionsmenge, ist.
      \item<3-> Angenommen, $\dist(R, \conv P) = 0$.
        Dann gibt es $r = \lambda_1 q_1 + ... + \lambda_k q_k \in R$ mit $\dist(r, \conv P) = 0$, also $r \in \conv P$.
      \item<4-> Dann können aber $q_1, ..., q_k$ nicht mit einer Hyperebene stark von $\conv P$ getrennt werden. Widerspruch.\\
      \item<5-> Für alle Mengen $T$ wie oben gilt dann $\conv T \subset R$ und somit $\dist(\conv T, \conv P) \geq \dist(R, \conv P)$.\\
      \item<6-> Durch Übergang zum Infimum folgt $\delta \geq \dist(R, \conv P) > 0$.
    \end{enumerate}
    \vspace{-16pt}
  \end{proof}
\end{frame}

\begin{frame}
  \begin{center}
    \scalebox{0.8}{
      \begin{tikzpicture}
        \path[use as bounding box] (-5,-2.5) rectangle (5,2.5);

        \draw [color=DarkGreen] (0.5,0)++(90:1.5) arc (90:270:1.5) arc (-90:90:0.5) arc (270:90:0.5) arc (-90:90:0.5) -- cycle;
        \fill [color=DarkGreen,opacity=0.20] (0.5,0)++(90:1.5) arc (90:270:1.5) arc (-90:90:0.5) arc (270:90:0.5) arc (-90:90:0.5) -- cycle;
        \fill [color=DarkGreen,opacity=0.20] (1,1) arc (0:-90:0.5) arc (90:270:0.5) arc(90:0:0.5) -- cycle;
        \node (P) [color=DarkGreen] at (-0.5,0) {P};

        %\begin{scope}[cm={cos(-15),-sin(-15),sin(-15),cos(-15), (0,0)}]
        %  \fill [color=DarkGreen,opacity=0.2] (-10,-1.6) -- (10,-1.6) -- (10,1.38) -- (-10, 1.38) -- cycle;
        %\end{scope}
        %\node [color=DarkGreen] at (-3,-1) {$\conv P + F$};

        \draw<3-> (-4, 0.25) circle (1);
        \node<3-> at (-5, 1.25) {$\Omega$};
        \draw<4-> (-4, 0.25) ++ (15:0.25) arc (15:105:0.25);
        \node<4-> [point,inner sep=0.5pt] at (-3.93, 0.38) {};
        \draw<4-> [->] (-4,0.25) -- + (15:1.25) node[anchor=west] {$r_w$};
        \draw<4-> (-4,0.25) -- + (105:1.25) node[anchor=south] {$F_w$};
        \draw<4-> (-4,0.25) -- + (-75:1.25);

        \begin{scope}[cm={cos(75),-sin(75),sin(75),cos(75), (0,0)}]
          \fill<5-> [color=blue,opacity=0.2] (-20,-1.03) -- (20,-1.03) -- (20,1.25) -- (-20, 1.25) -- cycle;
          \fill<7-> [color=yellow,opacity=0.3] (20,1.25) -- (-20,1.25) -- (-20,20) -- (20,20) -- cycle;
        \end{scope}
        \node<5-> [color=blue] at (0.75,-2.25) {$(\conv P) + F_w$};

        \fill<6-> [color=red,opacity=0.20] (0.5,0) ++ (105:1.5) arc (105:270:1.5) arc (270:285:0.5) -- ++ (15:10) -- ++ (105:2.8) -- cycle;
        \node<6-> [color=red] at (3,0.8) {$(\conv P) + r_w$};

        \node<7-> at (4,-2) {$G_w$};
      \end{tikzpicture}
    }
  \end{center}

  \small

  \begin{proof}[Beweis von Theorem 9.5]
    \renewcommand{\qedsymbol}{}
    \begin{enumerate}
      \item<2->{Sei $Z_1 = (\conv P) + F_1$ ein $k$-Zylinder. Annahme: $Z_1 \cap Q \not= \emptyset$.}
      \item<3->{Setze $\Omega := S_1(0) \cap F_1 = \{ x \in F \mid \norm{x} = 1 \}$.}
      \item<4->{Für $w \in \Omega$ sei $F_w$ das orthogonale Komplement zu $\mathsf{span} \{ w \}$ in $F_1$, also $F_1 = \mathsf{span} \{ w \} \perp F_w$ und $r_w := \R_{\geq 0} {\cdot} w$ der Strahl durch $w$.}
      \item<5->{Für $w \in \Omega$ sei $G_w$ diejenige Komponente von $Z_1 \setminus ((\conv P) + F_w)$, die $(\conv P) + r_w$ schneidet.}
    \end{enumerate}
    \vspace{-16pt}
  \end{proof}
\end{frame}

\begin{frame}
  \begin{center}
    \scalebox{0.8}{
      \begin{tikzpicture}
        \path[use as bounding box] (-8,-3) rectangle (6,2);

        \draw [color=DarkGreen] (0.5,0)++(90:1.5) arc (90:270:1.5) arc (-90:90:0.5) arc (270:90:0.5) arc (-90:90:0.5) -- cycle;
        \fill [color=DarkGreen,opacity=0.20] (0.5,0)++(90:1.5) arc (90:270:1.5) arc (-90:90:0.5) arc (270:90:0.5) arc (-90:90:0.5) -- cycle;
        \fill [color=DarkGreen,opacity=0.20] (1,1) arc (0:-90:0.5) arc (90:270:0.5) arc(90:0:0.5) -- cycle;
        \node (P) [color=DarkGreen] at (-0.5,0) {P};

        \draw (-7, 0) circle (1);
        \node at (-7.5, 1.25) {$\Omega$};

        \begin{scope}[cm={cos(-27),-sin(-27),sin(-27),cos(-27), (0,0)}]
          \fill<3-> [color=blue,opacity=0.2] (-20,-1.64) -- (20,-1.64) -- (20,1.28) -- (-20, 1.28) -- cycle;
        \end{scope}
        \node<3-> [color=blue] at (-4,-2.25) {$(\conv P) + F_w$};

        \draw<3-> [->] (-7,0) -- + (297:1.25) node[anchor=west] {$r_w$};
        \draw<3-> (-7,0) -- + (387:1.25) node[anchor=south] {$F_w$};
        \draw<3-> (-7,0) -- + (207:1.25);
        
        \begin{scope}[cm={cos(27),-sin(27),sin(27),cos(27), (0,0)}]
          \fill<4-> [color=blue,opacity=0.2] (-20,-1.26) -- (20,-1.26) -- (20,1.64) -- (-20, 1.64) -- cycle;
        \end{scope}
        \node<4-> [color=blue] at (-4,2.25) {$(\conv P) + F_{w'}$};

        \draw<4-> [->] (-7,0) -- + (63:1.25) node[anchor=south] {$r_{w'}$};
        \draw<4-> (-7,0) -- + (153:1.25) node[anchor=east] {$F_{w'}$};
        \draw<4-> (-7,0) -- + (-27:1.25);

        \node (q) [point,label=0:$q$] at (4.96,0) {};
        \draw<2-> (q) [dashed] circle (0.6);
        \node<2-> at (5.86,-0.2) {$S_q$};

        \draw<5-> [color=orange,ultra thick] (-7,0) ++ (-63:1) arc (-63:63:1);
        \node<5-> [color=orange] at (-5.5,0) {$A_q$};
      \end{tikzpicture}
    }
  \end{center}

  \begin{proof}[Beweis von Theorem 9.5]
    \renewcommand{\qedsymbol}{}
    \begin{enumerate}
      \setcounter{enumi}{5}
      \item Für $q \in Q \cap Z_1$ setze
        \begin{flalign*}
          \uncover<2->{S_q &:= B_{\delta/2}(q) \cap Z_1 = \{ x \in Z_1 \mid \norm{x - q} < \delta/2 \}}&\\
          \uncover<3->{A_q &:= \{ w \in \Omega \mid S_q \subset G_w \}}&
        \end{flalign*}
      \item<6-> Man kann zeigen: Für alle $q \in Q \cap Z_1$ ist $A_q$ kompakt und stark konvex.
    \end{enumerate}
    \vspace{-16pt}
  \end{proof}
\end{frame}

\begin{frame}
  \begin{proof}[Beweis von Theorem 9.5]
    \renewcommand{\qedsymbol}{}
    \textbf{Behauptung:} Seien $q_1, ..., q_m \in Q \cap Z_1$ mit $1 \leq m \leq k$.\\
    Dann gilt $\bigcap_{i=1}^m A_{q_i} \not= \emptyset$.
    \begin{enumerate}
      %\setcounter{enumi}{7}
      \item<2-> Es gilt $\dist(\conv \{ q_1, ..., q_m \}, \conv P) \geq \delta$.
      \item<3-> Es folgt $\dist(\conv (S_{q_1} \cup ... \cup S_{q_m}), \conv P) > \delta/2$.
      \item<4-> Also gibt es eine Hyperebene $H$, die $S_{q_1} \cup ... \cup S_{q_m}$ und $P$ streng trennt. Sei $H'$ der zu $H$ parallele $(n{-}1)$-dimensionale Unterraum.
      \item<5-> Es gilt $F \not\subset H'$, da sonst $Z_1 \cap Q = \emptyset$.
      \item<6-> Somit ist $G := H' \cap F_1$ ein $(k{-}1)$-dimensionaler Unterraum.
      \item<7-> Dann liegt $S_{q_1} \cup ... \cup S_{q_m}$ in einer der beiden Komponenten von $Z_1 \setminus ((\conv P) + G)$.
      \item<8-> Wähle $w \in \Omega$, sodass $w \perp G$ und $S_{q_1} \cup ... \cup S_{q_m} \subset G_w$.
      \item<9-> Folglich gilt $w \in \bigcap_{i=1}^m A_{q_i}$.
    \end{enumerate}
    \vspace{-12pt}
  \end{proof}
\end{frame}

\begin{frame}
  \begin{proof}[Beweis von Theorem 9.5]
    Wir haben gesehen, dass $\{ A_q \mid q \in Q \cap Z_1 \}$ eine Familie kompakter, stark konvexer Mengen ist. Zusammen mit vorheriger Behauptung folgt aus Lemma 9.4:

    \vspace{2pt}

    \uncover<2->{Es gibt ein Paar von antipodalen Punkten $\{ y, {-}y \}$ in $\Omega$, sodass $\fa{q \in Z_1 \cap Q} A_q \cap \{ p, {-}p \} \not= \emptyset$.}
    \uncover<3->{Somit hat der $(k{-}1)$-Zylinder $Z_2 := (\conv P) + F_y \subset Z_1$ leeren Schnitt mit $Q$.\qedhere}
  \end{proof}
\end{frame}

\section{Trennung durch Parallelotope}

\begin{frame}
  \frametitle{Übersicht}
  \tableofcontents[currentsection]
\end{frame}

\begin{frame}
  \begin{definition}
    Sei $\beta = \{ b_1, b_2, ..., b_n \}$ eine Basis von $\E^n$. Sei $H_i$ für $i = 1, ..., n$ die Hyperebene $\mathsf{span} (b_1, ..., \widehat{b_i}, ..., b_n)$. Eine \emph{$\beta$-Box} ist ein Parallelotop, in dem jede Seite parallel zu einer Hyperebene $H_i$ ist.
  \end{definition}

  \begin{tikzpicture}
    \path[use as bounding box] (-5,-1) rectangle (5,3.75);

    \draw<2> [->] (-3,-0.5) -- + (10:1) node [anchor=west] {$b_1$};
    \draw<2> [->] (-3,-0.5) -- + (120:2) node [anchor=south] {$b_2$};
    \draw<2> [->] (-3,-0.5) -- + (70:1.5) node [anchor=south] {$b_2$};
    \node<2> at (-3,-1.25) {$\beta = \{ b_1, b_2, b_3 \}$};

    \draw<2> (0.5,-0.5) -- ++ (10:4) -- ++ (120:1.5) -- ++ (10:-4) -- cycle;
    \draw<2> [dashed] (0.5,-0.5) ++ (70:2) -- ++ (10:4);
    \draw<2> [dashed] (0.5,-0.5) ++ (70:2) -- ++ (120:1.5);
    \draw<2> (0.5,-0.5) ++ (70:2) ++ (10:4) -- ++ (120:1.5) -- ++ (10:-4);
    \draw<2> [dashed] (0.5,-0.5) -- ++ (70:2);
    \draw<2> (0.5,-0.5) ++ (120:1.5) -- ++ (70:2);
    \draw<2> (0.5,-0.5) ++ (10:4) -- ++ (70:2);
    \draw<2> (0.5,-0.5) ++ (10:4) ++ (120:1.5) -- ++ (70:2);

    \fill<2> [color=yellow,opacity=0.35] (0.5,-0.5) -- ++ (10:4) -- ++ (120:1.5) -- ++ (10:-4) -- cycle;
    \fill<2> [color=yellow,opacity=0.2] (0.5,-0.5) ++ (70:2) -- ++ (10:4) -- ++ (120:1.5) -- ++ (10:-4) -- cycle;
    \fill<2> [color=yellow,opacity=0.2] (0.5,-0.5) -- ++ (10:4) -- ++ (70:2) -- ++ (10:-4) -- cycle;
    \fill<2> [color=yellow,opacity=0.2] (0.5,-0.5) ++ (120:1.5) -- ++ (10:4) -- ++ (70:2) -- ++ (10:-4) -- cycle;
    \fill<2> [color=yellow,opacity=0.2] (0.5,-0.5) -- ++ (70:2) -- ++ (120:1.5) -- ++ (70:-2) -- cycle;
    \fill<2> [color=yellow,opacity=0.4] (0.5,-0.5) ++ (10:4) -- ++ (70:2) -- ++ (120:1.5) -- ++ (70:-2) -- cycle;

    \node<2> at (2.5,-1.25) {Eine $\beta$-Box};
  \end{tikzpicture}
\end{frame}

\begin{frame}
  Die Koordinatenfunktionen dieser Basis sind
  \[
    \pi_i : \E^n \to \R, \quad
    \sum_{j=1}^n \lambda_j b_j \mapsto \lambda_i \qquad
    \text{für $i = 1, ..., n$.}
  \]
  Dann ist eine $\beta$-Box gegeben durch reelle Zahlen $m_1, ..., m_n$ und $M_1, ..., M_n$ mit $m_i \leq M_i$ für $i = 1, ..., n$ und besteht aus allen $x \in \E^n$, die folgendes lineare Ungleichungssystem erfüllen:

  \begin{align*}
    m_1 \leq &\pi_1(x) \leq M_1 \\
    m_2 \leq &\pi_2(x) \leq M_2 \\
    &\vdots \\
    m_n \leq &\pi_n(x) \leq M_n \\
  \end{align*}
\end{frame}

\begin{frame}
  Sei $P \subset \E^n$ nichtleer und kompakt. Dann existiert eine eindeutige minimale $\beta$-Box $B_P$, die $P$ enthält. Diese ist gegeben durch
  \[
    m_i := \inf_{p \in P} \pi_i(p)
    \quad \text{und} \quad
    M_i := \sup_{p \in P} \pi_i(p)
    \qquad \text{für $i = 1, ..., n$.}
  \]
  \begin{center}
    \begin{tikzpicture}
      \path[use as bounding box] (-6,-2) rectangle (2,2);

      \draw [->] (-5,-1.5) -- + (15:1) node [anchor=west] {$b_1$};
      \draw [->] (-5,-1.5) -- + (110:1.2) node [anchor=south] {$b_2$};
      \node at (-5,-2) {$\beta = \{ b_1, b_2 \}$};

      \draw [color=DarkGreen] (0,1.6) arc (90:240:0.7) arc (-300:0:0.7) arc (180:480:0.7) arc (-60:90:0.7);
      \fill [color=DarkGreen,opacity=0.20] (0,1.6) arc (90:240:0.7) arc (-300:0:0.7) arc (180:480:0.7) arc (-60:90:0.7);
      \node [color=DarkGreen] at (0,0) {$P$};

      \draw<2> (-1.02,-1.51) -- ++ (15:2.755) -- ++ (110:2.79) -- ++ (15:-2.755) -- cycle;
      \begin{scope}
        \clip (0,1.6) arc (90:240:0.7) arc (-300:0:0.7) arc (180:480:0.7) arc (-60:90:0.7) (-5,-5) rectangle (5,5);
        \fill<2> [color=yellow,opacity=0.4] (-1.02,-1.51) -- ++ (15:2.755) -- ++ (110:2.79) -- ++ (15:-2.755) -- cycle;
      \end{scope}
      \node at (0,-2) {minimale $\beta$-Box um $P$};
    \end{tikzpicture}
  \end{center}
\end{frame}

\begin{frame}
  \begin{theorem}[10.2]
    Seien $P$ und $Q$ nichtleere, kompakte Teilmengen von $\E^n$ ($n \geq 2$).\\
    Dann sind für eine Basis $\beta$ von $\E^n$ äquivalent:
    \begin{enumerate}[(a)] %[label={(\alph*)}]
      \item Es gibt eine $\beta$-Box $B$, sodass $P \subset B$ und $Q \cap B = \emptyset$.
      \item Für jede Teilmenge $S \subset P {\cup} Q$ mit maximal $n{+}1$ Punkten gibt es eine $\beta$-Box $B_S$, sodass $(P \cap S) \subset B_S$ und $(Q \cap S) \cap B_S = \emptyset$.
      \item Für jede Teilmenge $T \subset P$ mit maximal $n$ Punkten ist die minimale $\beta$-Box $B_T$, die $T$ enthält, disjunkt von $Q$, also $B_T \cap Q = \emptyset$.
    \end{enumerate}
  \end{theorem}

  \begin{proof}<2->
    \renewcommand{\qedsymbol}{}
    $(a) \Rightarrow (b)$ Klar.\\
    % Beweis aus dem Buch:
    %$(b) \Rightarrow (c)$
      %Angenommen, $(c)$ ist falsch. Dann gibt es eine Menge $T \subset P$ mit maximal $n$ Punkten, sodass $B_T \cap Q \not= \emptyset$. Sei $q \in B_T \cap Q$. Dann ist $S := T \cup \{ q \}$ eine Teilmenge von $P \cup Q$, die maximal $n {+} 1$ Punkte enthält. Jede $\beta$-Box, die $S \cap P = T$ enthält, enthält dann auch $q \in S \cap Q$. Damit stimmt $(b)$ nicht.
    % Besserer, da direkter Beweis:
    \visible<3->{$(b) \Rightarrow (c)$}
      \visible<3->{Sei $T \subset P$ eine Menge mit maximal $n$ Punkten und $B_T$ die minimale $\beta$-Box, die $T$ enthält.}
      \visible<4->{Sei $q \in Q$ beliebig.}
      \visible<5->{Dann ist die Menge $S_q := T \cup \{ q \}$ eine Teilmenge von $P \cup Q$ mit maximal $n {+} 1$ Punkten.}
      \visible<6->{Sei $B_{S_q}$ die $\beta$-Box aus $(b)$.}
      \visible<7->{Dann gilt: $T \subset P \cap S_q \subset B_{S_q}$, also $B_T \subset B_{S_q}$, und $B_T \cap \{ q \} \subset B_{S_q} \cap (S_q \cap Q) = \emptyset$.}
  \end{proof}
\end{frame}

\begin{frame}
  \begin{proof}
    \renewcommand{\qedsymbol}{}
    $(c) \Rightarrow (a)$ Durch Induktion über $n$.\\[2pt]
    \textbf{Induktionsanfang} ($n=2$):\\
    \uncover<2->{Sei $B_P$ die minimale $\beta$-Box, die $P$ enthält.}
    \uncover<3->{Da $P$ kompakt ist, können wir einen Punkt aus $P$ auf jeder der vier Seiten des Parallelogramms $B_P$ wählen. Nenne diese Punkte $p_1, p_2, p_3, p_4$.}
    \uncover<4->{Für jedes Paar von Punkten $p_i$ und $p_j$ mit $i \not= j$ sei $B_{ij}$ die minimale $\beta$-Box, die $p_i$ und $p_j$ enthält.}
    \uncover<5->{Es gilt $B_P = \bigcup_{i \not= j} B_{ij}$.}
    \uncover<6->{Angenommen, $(a)$ ist falsch, also $q \in Q \cap B_P = Q \cap ( \bigcup_{i \not= j} B_{ij} )$.}
    \uncover<7->{Dann gibt es $i, j \in \{ 1, 2, 3, 4 \}$ mit $i \not= j$ und $q \in B_{ij}$.}
  \end{proof}

  \begin{center}
    \scalebox{0.95}{
      \begin{tikzpicture}
        \path[use as bounding box] (-2,-2) rectangle (2,2.1);

        \draw<1-3> [color=DarkGreen] (0,1.6) arc (90:240:0.7) arc (-300:0:0.7) arc (180:480:0.7) arc (-60:90:0.7);
        \fill<1-3> [color=DarkGreen,opacity=0.20] (0,1.6) arc (90:240:0.7) arc (-300:0:0.7) arc (180:480:0.7) arc (-60:90:0.7);
        \node<1-3> [color=DarkGreen] at (0,0) {$P$};

        \draw<2-> (-1.02,-1.51) -- ++ (15:2.755) -- ++ (110:2.79) -- ++ (15:-2.755) -- cycle;
        \begin{scope}
          \clip (0,1.6) arc (90:240:0.7) arc (-300:0:0.7) arc (180:480:0.7) arc (-60:90:0.7) (-5,-5) rectangle (5,5);
          \fill<2-> [color=yellow,opacity=0.4] (-1.02,-1.51) -- ++ (15:2.755) -- ++ (110:2.79) -- ++ (15:-2.755) -- cycle;
        \end{scope}

        \node<2-> at (-1.5,1.7) {$B_P$};

        \node<3-> (p1) [point,label=270:$p_1$] at (0.9,-1) {};
        \node<3-> (p2) [point,label=0:$p_2$] at (1.38,-0.1) {};
        \node<3-> (p3) [point,label=90:$p_3$] at (-0.2,1.6) {};
        \node<3-> (p4) [point,label=180:$p_4$] at (-1.37,-0.55) {};

        \draw<4-> (p1) -- ++ (110:2.79);
        \draw<4-> (p2) -- ++ (15:-2.755);
        \draw<4-> (p3) -- ++ (110:-2.79);
        \draw<4-> (p4) -- ++ (15:2.755);
        \node<4-> at (-0.7,0.6) {$B_{34}$};
      \end{tikzpicture}
    }
  \end{center}
\end{frame}

\begin{frame}
  \begin{proof}
    \renewcommand{\qedsymbol}{}
    $(c) \Rightarrow (a)$ Durch Induktion über $n$.\\[2pt]
    \textbf{Induktionsschritt} ($n \to n {+} 1$):\\
    \uncover<2->{Sei $B_P$ die minimale $\beta$-Box, die $P$ enthält.}
    \uncover<3->{Angenommen, $(a)$ gilt nicht, es gibt also $q \in B_P \cap Q$.}
    \uncover<4->{
      Definiere $f : \E^{n+1} \to E^n$ durch\\
      $f(\lambda_1 b_1 + ... + \lambda_{n+1} b_n) = \lambda_1 b_1 + ... + \lambda_n b_n$.
    }
    \uncover<5->{Dann ist $\beta' := \{ b_1, ..., b_n \}$ eine Basis von $\E^n$ und $f(B_P)$ die minimale $\beta'$-Box, die $f(P)$ enthält.}
    \uncover<6->{Da $f(q) \in f(B_P)$, gibt es nach Induktionsannahme eine Teilmenge $T' \subset f(P)$ mit maximal $n {-} 1$ Punkten, sodass $f(q)$ in der minimalen $\beta'$-Box um $T'$ enthalten ist.}
    \uncover<7->{Sei $T \subset P$ mit maximal $n {-} 1$ Punkten und $f(T) = T'$.}
    \uncover<8->{
      Es gilt
      \begin{align*}
        \inf_{x \in T} \pi_i(x) \leq \pi_i(q) \leq \sup_{x \in T} \pi_i(x)
        \qquad \text{für } i \in \{1, ..., n\}.
      \end{align*}
    }
    \uncover<9->{Angenommen, obige Ungleichung gilt auch für $i = n {+} 1$. Dann sind wir fertig.}
    \uncover<10->{
      Andernfalls gilt
      \[
        \pi_{n+1}(q) > \sup_{x \in T} \pi_{n+1}(x)
        \quad \text{oder} \quad
        \pi_{n+1}(q) < \inf_{x \in T} \pi_{n+1}(x).
      \]
    }
  \end{proof}
\end{frame}

\begin{frame}
  \begin{proof}
    $(c) \Rightarrow (a)$ Durch Induktion über $n$.\\[2pt]
    \textbf{Induktionsschritt} (Fortsetzung):\\
    Angenommen, es gilt $\pi_{n+1}(q) > \sup_{x \in T} \pi_{n+1}(x)$.
    \uncover<2->{Da $P$ kompakt ist, gibt es $p \in P$ mit $\pi_{n+1}(p) = \sup_{x \in P} \pi_{n+1}(x)$.}
    \uncover<3->{Aus $q \in B_P$ folgt $\pi_{n+1}(q) \leq \pi_{n+1}(p)$.}
    \uncover<4->{
      Somit gilt für $i \in \{ 1, ..., n, n {+} 1 \}$:
      \[
        \inf_{x \in T \cup \{ p \}} \pi_i(x) \enspace\leq\enspace \pi_i(q) \enspace\leq \sup_{x \in T \cup \{ p \}} \pi_i(x)
      \]
    }
    \uncover<5->{Widerspruch zu $(c)$.}
    \uncover<6->{Der andere Fall folgt analog.\qedhere}
  \end{proof}

  \begin{center}
    \scalebox{0.88}{
      \begin{tikzpicture}
        \path[use as bounding box] (-3,-2) rectangle (3,1.75);

        \draw [->,gray] (-3.5,0,1) -- ++ (0,1,0) node [anchor=south] {$b_{n+1}$};
        \draw [->,gray] (-3.5,0,1) -- ++ (1,0,0);
        \draw [->,gray] (-3.5,0,1) -- ++ (0,0,1.5);

        \draw (-1,0,1) -- ++ (2,0,0) -- ++ (0,0,-1.6);
        \draw [dashed] (-1,0,1) -- ++ (0,0,-1.6) -- ++ (2,0,0);

        \draw (-1,0,1) ++ (0,0.8,0) -- ++ (2,0,0) -- ++ (0,0,-1.6);
        \draw [dashed] (-1,0,1) ++ (0,0.8,0) -- ++ (0,0,-1.6) -- ++ (2,0,0);

        \draw<2-> (-1,0,1) ++ (0,1.6,0) -- ++ (2,0,0) -- ++ (0,0,-1.6);
        \draw<2-> (-1,0,1) ++ (0,1.6,0) -- ++ (0,0,-1.6) -- ++ (2,0,0);

        \draw (-1,0,1) -- ++ (0,0.8,0);
        \draw (-1,0,1) ++ (2,0,0) -- ++ (0,0.8,0);
        \draw [dashed] (-1,0,1) ++ (0,0,-1.6) -- ++ (0,0.8,0);
        \draw (-1,0,1) ++ (2,0,-1.6) -- ++ (0,0.8,0);

        \draw<2-> (-1,0,1) ++ (0,0.8,0) -- ++ (0,0.8,0);
        \draw<2-> (-1,0,1) ++ (0,0.8,0) ++ (2,0,0) -- ++ (0,0.8,0);
        \draw<2-> [dashed] (-1,0,1) ++ (0,0.8,0) ++ (0,0,-1.6) -- ++ (0,0.8,0);
        \draw<2-> (-1,0,1) ++ (0,0.8,0) ++ (2,0,-1.6) -- ++ (0,0.8,0);

        \draw (-1,0,1) ++ (0,-1.6,0) -- ++ (2,0,0) -- ++ (0,0,-1.6);
        \draw (-1,0,1) ++ (0,-1.6,0) -- ++ (0,0,-1.6) -- ++ (2,0,0);

        \draw [decorate,decoration={brace,amplitude=4pt},xshift=-4pt] (-1,0,1) -- ++ (0,0.8,0) node [black,midway,xshift=-0.4cm] {\footnotesize $B_T$};

        \node<2-> (p) [point,label={[label distance=-2pt]0:$p$}] at (-0.6,1.6,0) {};
        \node (q) [point,label={[label distance=-2pt]0:$q$}] at (0.8,1.2,-0.2) {};

        \draw [blue,thick] (q) -- (0.8,0.8,-0.2);
        \draw [blue,thick] (0.8,0.6,-0.2) -- (0.8,0,-0.2);
        \draw [blue,thick] (0.8,-0.2,-0.2) -- (0.8,-1.6,-0.2);

        \draw [->,thick,orange] (0.1,-0.15,1) -- ++ (0,-0.7,0);
        \node (f) at (-0.1,-0.5,1) {$f$};

        \node (fq) [point,label={[label distance=4pt]0:$f(q)$}] at (0.8,-1.6,-0.2) {};

        \node at (0.1,-1.9,1) {$B_{T'}$};
      \end{tikzpicture}
    }
  \end{center}
\end{frame}

\begin{frame}
  \begin{block}{Korollar (10.3)}
    Sei $P \subset \E^n$ ($n \geq 2$) nichtleer und kompakt.\\
    Angenommen, $q \in \E^n$ erfüllt das System von Ungleichungen
    \[
      \inf_{x \in P} \pi_i(x) \,\leq\, \pi_i(q) \,\leq\, \sup_{x \in P} \pi_i(x)
      \quad \text{für $i \in \{ 1, ..., n \}$.}
    \]
    Dann gibt es eine Menge $T \subset P$ mit höchstens $n$ Punkten und
    \[
      \inf_{x \in T} \pi_i(x) \,\leq\, \pi_i(q) \,\leq\, \sup_{x \in T} \pi_i(x)
      \quad \text{für $i \in \{ 1, ..., n \}$.}
    \]
  \end{block}
\end{frame}

\begin{frame}
  \begin{theorem}[Carathéodory, 2.23]
    Sei $P \subset \E^n$ nichtleer und sei $q \in \E^n$.\\
    Dann ist jeder Punkt $p \in P$ eine Konvexkombination von maximal $n{+}1$ Punkten aus $P$.
  \end{theorem}

  \begin{theorem}[Carathéodory, Umformulierung]
    Sei $P \subset \E^n$ nichtleer und sei $q \in \E^n$.\\
    Angenommen, für jede Menge $T \subset P$ mit höchstens $n{+}1$ Punkten gibt es eine Hyperebene, die $T$ und $\{ q \}$ streng trennt. Dann ist $q$ nicht in der konvexen Hülle von $P$ enthalten.
  \end{theorem}
\end{frame}

\begin{frame}
  \begin{block}{Korollar (10.4)}
    Sei $P \subset \E^n$ \alert<2->{($n \geq 2$)} nichtleer und \alert<2->{kompakt} und sei $q \in \E^n$.\\
    Sei $\beta = \{ b_1, ..., b_n \}$ eine Basis von $\E^n$ und $H_i$ der von $\beta \setminus \{ b_i \}$ aufgespannte Unterraum für $i \in \{ 1, ..., n \}$. Angenommen, für jede Menge $T \subset P$ mit höchstens \alert<2->{$n$} Punkten gibt es eine Hyperebene, die \alert<2->{parallel} zu einem der $H_i$ ist und $T$ und $\{ q \}$ streng trennt. Dann ist $P$ nicht in der konvexen Hülle von $P$ enthalten.
  \end{block}

  \begin{proof}<3->
    Wegen Satz 10.4 gibt es eine $\beta$-Box $B_P$, die $P$ enthält und disjunkt zu $\{ q \}$ ist.
    \uncover<4->{Es folgt $\{ q \} \cap \conv P \subset \{ q \} \cap B_P = \emptyset$, also $q \not\in \conv P$.\qedhere}
  \end{proof}
\end{frame}

\begin{frame}[plain]
  \begin{center}
    \huge Danke für die Aufmerksamkeit!
  \end{center}
\end{frame}

\end{document}