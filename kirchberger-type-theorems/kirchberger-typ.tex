\documentclass{beamer}

\usepackage[utf8]{inputenc}
\usepackage[ngerman]{babel}

\usepackage{tikz}
\usetikzlibrary{3d,calc}
\usepackage{fix-cm}
\usepackage{tikz-3dplot}

% http://stackoverflow.com/questions/1670463/latex-change-margins-of-only-a-few-pages
\newenvironment{changemargin}[2]{%
  \begin{list}{}{%
    \setlength{\topsep}{0pt}%
    \setlength{\leftmargin}{#1}%
    \setlength{\rightmargin}{#2}%
    \setlength{\listparindent}{\parindent}%
    \setlength{\itemindent}{\parindent}%
    \setlength{\parsep}{\parskip}%
  }%
  \item[]}{
  \end{list}
}

\newtranslation[to=ngerman]{Example}{Beispiel}

\usetheme{Frankfurt}
%\beamertemplatenavigationsymbolsempty
%\setbeamertemplate{mini frames}{}
\useoutertheme[subsection = false]{miniframes}
\usepackage{color,graphicx,overpic}
\definecolor{DarkBlue}{rgb}{0.098,0.098,0.349} % RGB(25, 25, 89)
\setbeamercolor{section in head/foot}{fg=white, bg=DarkBlue}
\definecolor{LighterBlue}{rgb}{0.15,0.2,0.8}
\setbeamercolor{frametitle}{bg=LighterBlue, fg=white}
\setbeamercolor{title}{fg=white, bg=LighterBlue}
\setbeamercolor{structure}{fg=LighterBlue, bg=white}

\definecolor{DarkGreen}{rgb}{0.15,0.7,0.15}

\newcommand{\E}{\mathrm{E}} % Euklidischer Raum
\newcommand{\R}{\mathbb{R}} % Reelle Zahlen
\newcommand{\fa}[1]{\forall \, {#1} \,:\,}
\newcommand{\ex}[1]{\exists \, {#1} \,:\,}
\usepackage{mathtools}
\DeclarePairedDelimiter\norm{\lVert}{\rVert}%

% http://tex.stackexchange.com/questions/48381/line-through-two-points-with-offset-in-tikz
\tikzset{
    add/.style args={#1 and #2}{
        to path={%
  ($(\tikztostart)!-#1!(\tikztotarget)$)--($(\tikztotarget)!-#2!(\tikztostart)$)%
  \tikztonodes},add/.default={.2 and .2}}
}

% http://tex.stackexchange.com/questions/6135/how-to-make-beamer-overlays-with-tikz-node
\tikzset{onslide/.code args={<#1>#2}{
  \only<#1>{\pgfkeysalso{#2}} % \pgfkeysalso doesn't change the path
}}
\tikzset{temporal/.code args={<#1>#2#3#4}{
  \temporal<#1>{\pgfkeysalso{#2}}{\pgfkeysalso{#3}}{\pgfkeysalso{#4}} % \pgfkeysalso doesn't change the path
}}

\tikzstyle{point}=[circle,fill,inner sep=1pt]
\tikzstyle{cover}=[LightGray]

\definecolor{LightGray}{rgb}{0.8,0.8,0.8}
\newcommand<>{\cover}[1]{\alt#2{{\color{LightGray} {#1}}}{#1}}

% s/Theorem/Satz ?
% Kirchberger-ähnliche Theoreme?
% Theoreme vom Kirchberger-Typ?
% Varianten des Theorems von Kirchberger
\title{Varianten des Theorems von Kirchberger}
\author{Tim Baumann}
\institute{TopMath-Frühlingsschule in Oberschönenfeld}
\date{4. März 2014}

\begin{document}

\begin{frame}[plain]
  \titlepage
\end{frame}

\begin{frame}[plain]
  \begin{theorem}[Kirchberger]
    % TODO: trennbar oder streng trennbar?
    Seien $P$ und $Q$ nichtleere, kompakte Teilmengen von $\E^n$. \\
    Dann sind $P$ und $Q$ genau dann durch eine \alert<2>{Hyperebene} trennbar, wenn für jede Menge $T \subset \E^n$ mit maximal $n + 2$ Punkten die Mengen $P \cap T$ und $Q \cap T$ durch eine \alert<2>{Hyperebene} trennbar sind.
  \end{theorem}
\end{frame}

\section{Trennung durch Sphären}
\begin{frame}
  \frametitle{Übersicht}
  \tableofcontents[currentsection]
\end{frame}

%\iffalse
\begin{frame}
  %\frametitle{Sphären}
  \begin{definition}
    Sei $p \in \E^n$ und $\alpha > 0$. Dann heißt
    \[ S_\alpha(p) := \left\{ x \in \E^n \mid \norm{x - p} = \alpha \right\} \]
    \emph{Sphäre} mit Radius $\alpha$ um den Punkt $p$.
  \end{definition}
  \begin{definition}<2->
    Seien $A$ und $B$ Teilmengen von $\E^n$.\\
    Die Sphäre $S_\alpha(p)$ \emph{trennt} $A$ und $B$ \emph{streng}, wenn gilt:
    \begin{minipage}{0.39 \linewidth}
      \begin{uncoverenv}<3->
        \vspace{10pt}
        \begin{center}
          \colorbox{white}{
            \begin{tikzpicture}
              \node (A) at (0, 0) {A};
              \draw (0, 0) circle (0.4);
              \node (B) at (1, 0) {B};
            \end{tikzpicture}
          }
        \end{center}
      \end{uncoverenv}
      \begin{uncoverenv}<1->
        \[ \fa{a \in A} \norm{p - a} < \alpha \]
        \vspace{-30pt}

        \begin{center}
          und
        \end{center}
        \vspace{-34pt}

        \[ \fa{b \in B} \norm{p - a} > \alpha \]
        \vspace{-20pt}
      \end{uncoverenv}
    \end{minipage}
    \begin{minipage}{0.19 \linewidth}
      \begin{uncoverenv}<4->
        \vspace{8pt}
        \begin{center}
          oder
        \end{center}
      \end{uncoverenv}
    \end{minipage}
    \begin{minipage}{0.39 \linewidth}
      \begin{uncoverenv}<4->
        \vspace{10pt}
        \begin{center}
          \colorbox{white}{
            \begin{tikzpicture}
              \node (A) at (0, 0) {A};
              \draw (1, 0) circle (0.4);
              \node (B) at (1, 0) {B};
            \end{tikzpicture}
          }
        \end{center}
      \end{uncoverenv}
      \begin{uncoverenv}<4->
        \[ \fa{a \in A} \norm{p - a} > \alpha \]
        \vspace{-30pt}

        \begin{center}
          und
        \end{center}
        \vspace{-34pt}

        \[ \fa{b \in B} \norm{p - a} < \alpha \]
        \vspace{-20pt}
      \end{uncoverenv}
    \end{minipage}
  \end{definition}
\end{frame}

\begin{frame}
  \begin{theorem}[Kirchberger\only<2->{'}]
    % TODO: trennbar oder streng trennbar?
    Seien $P$ und $Q$ nichtleere, kompakte Teilmengen von $\E^n$. \\
    Dann sind $P$ und $Q$ genau dann durch eine \alt<2->{\alert<2>{Sphäre}}{Hyperebene} streng trennbar,
    wenn für jede Menge $T \subset \E^n$ mit maximal \alt<3->{\alert<3->{$n + 3$}}{$n + 2$} Punkten die Mengen $P \cap T$ und $Q \cap T$ durch eine \alt<2->{\alert<2>{Sphäre}}{Hyperebene}\\
    streng trennbar sind.
  \end{theorem}
\end{frame}

\begin{frame}
  Folgendes Beispiel im $\E^2$ zeigt, dass die Trennbarkeit von $n + 2 = 4$ Punkten aus $P = \{ p_1, p_2, p_3 \}$ und $Q = \{ q_1, q_2 \}$ mit Sphären nicht ausreicht, um Trennbarkeit von $P$ und $Q$ zu folgern:

  \begin{center}
    \begin{tikzpicture}
      \path[use as bounding box] (-2,2.5) rectangle (6,0);
      \draw<4> [gray,dashed,add= .5 and .5] (q1) to (q2);

      \node (p1) [point,DarkGreen,onslide=<5-6>{cover},label=180:\cover<5-6>{$p_1$}] at (0, 2) {};
      \node (p2) [point,DarkGreen,label=-90:$p_2$] at (0, -2) {};
      \node (p3) [point,DarkGreen,onslide=<7-8>{cover},label=-45:\cover<7-8>{$p_3$}] at (2, 0) {};
      \draw [LightGray] (p1) -- (p2);
      \draw [LightGray] (p2) -- (p3);
      \draw [LightGray] (p3) -- (p1);
      \node<2-> (q1) [violet,point,onslide=<9-10>{cover},label=225:\cover<9-10>{$q_1$}] at (0, 0) {};
      \node<3-> (q2) [violet,point,onslide=<11-12>{cover},label=45:\cover<11-12>{$q_2$}] at (4, 0) {};

      \draw<6,13> (2, -2) [blue,dashed] circle (2.25);
      \draw<8,13> (2, 0) [blue,dashed] circle (2.25);
      \draw<10,13> (4, 0) [blue,dashed] circle (1);
      \draw<12,13> (0, 0) [blue,dashed] circle (1);
    \end{tikzpicture}
  \end{center}
\end{frame}
%\fi

\begin{frame}
  \begin{theorem}[Kirchberger']
    % TODO: trennbar oder streng trennbar?
    Seien $P$ und $Q$ nichtleere, kompakte Teilmengen von $\E^n$. \\
    Dann sind $P$ und $Q$ genau dann durch eine Sphäre streng trennbar,
    wenn für jede Menge $T \subset \E^n$ mit maximal $n + 3$ Punkten die Mengen $P \cap T$ und $Q \cap T$ durch eine Sphäre\\
    streng trennbar sind.
  \end{theorem}
\end{frame}

\setlength{\leftmargin}{-20pt}

\begin{changemargin}{-0.5cm}{-0.5cm}

\begin{frame}
  \vspace{-16pt}
  \begin{center}
    \begin{tikzpicture}
      [grid/.style={very thin,LightGray},
      axis/.style={->,gray}]

      \path[use as bounding box] (0,1.5,0) -- (6.5,0,0) -- (0,0,7);

      % Draw a grid in the x-z plane
      \foreach \x in {0,0.5,...,5}{
        \draw[grid] (\x,0,-0.5) -- (\x,0,6.5);
      }
      \foreach \z in {0,0.5,...,6}{
        \draw[grid] (-0.5,0,\z) -- (5.5,0,\z);
      }

      \draw<2->[axis] (0,0,0) -- (0,1,0) node[anchor=east]{$\E$};
      \draw[axis] (0,0,0) -- (6,0,0) node[anchor=south]{$\E^n$};
      \draw[axis] (0,0,0) -- (0,0,6.5) node[anchor=north]{};

      \shade<3->[ball color=blue!20!white,opacity=0.30] (1.5,1,3) circle (1);

      \node<3-> (p) [point,label=225:\only<3>{$p$}] at (1.5, 0, 3) {};
      \node<3-> (n) [point] at (1.5, 2, 3) {};

      % Find stereographic projection of $x$ with Wolfram Alpha:
      % ||t*{1.5,2,3} + (1-t)*{3,0,5} - {1.5,1,3}|| = 1
      \node<4-5> (x) [point,label=225:$x$] at (3, 0, 5) {};
      \node<5-5> (x') [point,label=45:{$x'=\phi(x)$}] at (2.08537, 1.21951, 3.78049) {};
      \draw<5> [dashed] (x) -- (x');
      \draw<5> [dashed,gray] (x') -- (n);

      \node<6-> (P) [color=DarkGreen] at (1.5, 0, 5) {P};
      \node<6-> (Q) [color=violet] at (4, 0, 5) {Q};
      %\node<3-> (P) [label=225:\cover<9-10>{$p$}] at (1.5, 0, 3) {};

      % Find stereographic projection of $P$ with Wolfram Alpha:
      % ||t*{1.5,2,3} + (1-t)*{1.5,0,5} - {1.5,1,3}|| = 1
      \node<7-> (P') [color=DarkGreen,font=\scriptsize] at (1.5, 1, 4) {P'};
      \draw<7> [dashed] (P) -- (P');
      \draw<7> [dashed,gray] (P') -- (n);

      % Find stereographic projection of $Q$ with Wolfram Alpha:
      % ||t*{1.5,2,3} + (1-t)*{4,0,5} - {1.5,1,3}|| = 1
      \node<7-> (Q') [color=violet,font=\scriptsize] at (2.20176, 1.4386, 3.5614) {Q'};
      \draw<7> [dashed] (Q) -- (Q');
      \draw<7> [dashed,gray] (Q') -- (n);

      \begin{scope}[canvas is zx plane at y=0]
        \draw<10-13> (5, 1.5) circle (0.5);
      \end{scope}

      \begin{scope}[canvas is xy plane at z=3.94]
        \draw<11-13> (1.5, 1) circle (0.24);
        \begin{scope}
          \clip (1.5, 1) circle (0.24) (-5,-5) rectangle (5,5);
          \shade<12-13> [ball color=blue, opacity=0.25] (0.75,0.25) rectangle (2.25,1.75);
        \end{scope}
      \end{scope}

      \draw<11-13> [gray] (1, 0, 5.05) -- (n);
      \draw<11-13> [gray] (2, 0, 4.9) -- (n);

      \begin{scope}[canvas is xy plane at z=3.85]
        \draw<14-> (1.51, 1.05) circle (0.5);
        \begin{scope}
          \clip (1.51, 1.05) circle (0.5) (-5,-5) rectangle (5,5);
          \shade<14-> [ball color=orange, opacity=0.25] (0.75,0.25) rectangle (2.25,1.75);
        \end{scope}
      \end{scope}

      \draw<17-> [gray] (0.2, 0, 5.05) -- (n);
      \draw<17-> [gray] (2.55, 0, 4.65) -- (n);

      \begin{scope}[canvas is zx plane at y=0]
        \draw<17-> (5, 1.4) circle (1.2);
      \end{scope}
    \end{tikzpicture}
  \end{center}

  \vspace{-10pt}

  \only<1-7>{
    \begin{enumerate}
      \item<2-> Bette $\E^n$ wie üblich in den $\E^{n+1}$ ein.
      \item<3-> Sei $p \!\in\! \E^n$ und $S \!\subset\! \E^{n+1}$ eine Sphäre, die in $p$ tangential zu $\E^n$ ist.
      \item<4-> Betrachte die stereographische Projektion $\phi : \E^n \to S$.
      \item<6-> Seien ${\color{DarkGreen} P}, {\color{violet} Q} \subset \E^n$ nichtleer und kompakt sodass für jede Menge $T \subset \E^n$ mit maximal $n + 3$ Punkten die Mengen $P \cap T$ und $Q \cap T$ durch eine Sphäre streng trennbar sind.
      \item<7-> Seien ${\color{DarkGreen} P'}$ und ${\color{violet} Q'}$ die (kompakten) Bilder von ${\color{DarkGreen} P}$ bzw. ${\color{violet} Q}$ unter $\phi$.
    \end{enumerate}
  }

  \only<8-13>{
    \emph{Behauptung:} ${\color{DarkGreen} P'}$ und ${\color{violet} Q'}$ können durch eine Hyperebene ${\color{orange} H_0} \subset \E^{n+1}$ streng getrennt werden.
    \vspace{-4pt}

    \begin{enumerate}
      \setcounter{enumi}{5}
      \item<9-> Sei $T \subset S \subset \E^{n+1}$ eine Menge mit höchstens $n + 3$ Punkten.
      \item<10-> Nach Voraussetzung werden die Urbilder $\phi^{-1}(T \!\cap\! P') = \phi^{-1}(T) \cap P$ und $\phi^{-1}(T \cap Q') = \phi^{-1}(T) \cap Q$ durch eine Sphäre streng getrennt.
      \item<11-> Die stereogr. Projektion der Sphäre ist ein Kreis auf $S$ (Kreistreue).
      \item<12-> Der Kreis auf $S$ ist der Schnitt von $S$ mit einer Hyperebene $H$.
      \item<13-> Da $H$ dann $T \cap P'$ und $T \cap Q'$ streng trennt, folgt die Behauptung nach dem Satz von Kirchberger.
    \end{enumerate}
    \vspace{-22pt}
  }

  \only<14-17>{
    \begin{enumerate}
      \setcounter{enumi}{10}
      \item<14-> Sei $\alpha \in \E^{n+1}$ und $b \in \R$, sodass $\langle \alpha, p \rangle < b$ für alle $p \in P'$ und $\langle \alpha, q \rangle > b$ für alle $q \in Q'$.
      \item<15-> Da $P'$ und $Q'$ kompakt sind, gibt es $\epsilon > 0$ mit $\langle \alpha, p \rangle \leq b - \epsilon$ für alle $p \in P'$ und $\langle \alpha, q \rangle \geq b + \epsilon$ für alle $q \in Q'$.
      \item<16-> Somit können wir annehmen, dass $H_0$ den Nordpol der Sphäre $S$ nicht schneidet.
      \item<17-> Der Schnitt $H_0 \cap S$ ist ein Kreis und $\phi^{-1}(H_0 \cap S)$ trennt $P$ und $Q$.
    \end{enumerate}
  }
\end{frame}

\end{changemargin}

\section{Trennung durch Zylinder}

\begin{frame}
  \frametitle{Übersicht}
  \tableofcontents[currentsection]
\end{frame}

\begin{frame}
  \begin{definition}
    Sei $A \!\subset\! \E^n$ und $F \!\subset\! \E^n$ ein $k$-dimensionaler Unterraum. Dann heißt
    \[ Z = A + F = \{ a + f \mid a \in A, f \in F \} \]
    von $A$ und $F$ erzeugter \emph{$k$-Zylinder}.
  \end{definition}

  \begin{tikzpicture}
    \draw (0,0,1) -- (1,0,0) -- (0,0,-1) -- (-1,0,0) -- (0,0,1);
    \draw (0,1,0) -- (0,0,1) -- (0,-1,0);
    \draw (0,1,0) -- (1,0,0) -- (0,-1,0);
    \draw (0,1,0) -- (0,0,-1) -- (0,-1,0);
    \draw (0,1,0) -- (-1,0,0) -- (0,-1,0);

    \draw (-5,1,1) -- (-5,0,2) -- (-5,-1,1);
    \draw (-5,1,1) -- (-5,0,0) -- (-5,-1,1);

    \draw (5,1,-1) -- (5,0,0) -- (5,-1,-1);
    \draw (5,1,-1) -- (5,0,-2) -- (5,-1,-1);
  \end{tikzpicture}
\end{frame}

\section{Trennung durch Parallelotope}

\begin{frame}
  \frametitle{Übersicht}
  \tableofcontents[currentsection]
\end{frame}

\begin{frame}
  TODO
\end{frame}

\end{document}