\documentclass{beamer}

\usepackage[utf8]{inputenc}
\usepackage[ngerman]{babel}
\usepackage{tikz}
\usetikzlibrary{calc}
\usepackage{fix-cm}

\newtranslation[to=ngerman]{Example}{Beispiel}

\usetheme{Frankfurt}
\usepackage{color,graphicx,overpic}
\definecolor{DarkBlue}{rgb}{0.098,0.098,0.349} % RGB(25, 25, 89)
\setbeamercolor{section in head/foot}{fg=white, bg=DarkBlue}
\definecolor{LighterBlue}{rgb}{0.15,0.2,0.8}
\setbeamercolor{frametitle}{bg=LighterBlue, fg=white}
\setbeamercolor{title}{fg=white, bg=LighterBlue}
\setbeamercolor{structure}{fg=LighterBlue, bg=white}

\newcommand{\E}{\mathrm{E}}
\newcommand{\fa}[1]{\forall \, {#1} \,:\,}
\newcommand{\ex}[1]{\exists \, {#1} \,:\,}
\usepackage{mathtools}
\DeclarePairedDelimiter\norm{\lVert}{\rVert}%

% http://tex.stackexchange.com/questions/48381/line-through-two-points-with-offset-in-tikz
\tikzset{%
    add/.style args={#1 and #2}{
        to path={%
 ($(\tikztostart)!-#1!(\tikztotarget)$)--($(\tikztotarget)!-#2!(\tikztostart)$)%
  \tikztonodes},add/.default={.2 and .2}}
}  

% s/Theorem/Satz ?
% Kirchberger-ähnliche Theoreme?
% Theoreme vom Kirchberger-Typ?
% Varianten des Theorems von Kirchberger
\title{Varianten des Theorems von Kirchberger}
\author{Tim Baumann}
\institute{TopMath-Frühlingsschule in Oberschönenfeld}
\date{4. März 2014}

\begin{document}

\begin{frame}[plain]
  \titlepage
\end{frame}

\begin{frame}[plain]
  \begin{theorem}[Kirchberger]
    % TODO: trennbar oder streng trennbar?
    Seien $P$ und $Q$ nichtleere, kompakte Teilmengen von $\E^n$. \\
    Dann sind $P$ und $Q$ genau dann durch eine \alert<2>{Hyperebene} trennbar, wenn für jede Menge $T \subset \E^n$ mit maximal $n + 2$ Punkten die Mengen $P \cap T$ und $Q \cap T$ durch eine \alert<2>{Hyperebene} trennbar sind.
  \end{theorem}
\end{frame}

\section{Trennung durch Sphären}
\begin{frame}
  \frametitle{Übersicht}
  \tableofcontents[currentsection]
\end{frame}

\begin{frame}
  %\frametitle{Sphären}
  \begin{definition}
    Sei $p \in \E^n$ und $\alpha > 0$. Dann heißt
    \[ S_\alpha(p) := \left\{ x \in \E^n \mid \norm{x - p} = \alpha \right\} \]
    \emph{Sphäre} mit Radius $\alpha$ um den Punkt $p$.
  \end{definition}
  \begin{definition}<2->
    Seien $A$ und $B$ Teilmengen von $\E^n$.\\
    Die Sphäre $S_\alpha(p)$ \emph{trennt} $A$ und $B$ \emph{streng}, wenn gilt:
    \begin{minipage}{0.39 \linewidth}
      \begin{uncoverenv}<3->
        \vspace{10pt}
        \begin{center}
          \colorbox{white}{
            \begin{tikzpicture}
              \node (A) at (0, 0) {A};
              \draw (0, 0) circle (10pt);
              \node (B) at (1, 0) {B};
            \end{tikzpicture}
          }
        \end{center}
      \end{uncoverenv}
      \begin{uncoverenv}<1->
        \[ \fa{a \in A} \norm{p - a} < \alpha \]
        \vspace{-30pt}

        \begin{center}
          und
        \end{center}
        \vspace{-34pt}

        \[ \fa{b \in B} \norm{p - a} > \alpha \]
        \vspace{-20pt}
      \end{uncoverenv}
    \end{minipage}
    \begin{minipage}{0.19 \linewidth}
      \begin{uncoverenv}<4->
        \vspace{8pt}
        \begin{center}
          oder
        \end{center}
      \end{uncoverenv}
    \end{minipage}
    \begin{minipage}{0.39 \linewidth}
      \begin{uncoverenv}<4->
        \vspace{10pt}
        \begin{center}
          \colorbox{white}{
            \begin{tikzpicture}
              \node (A) at (0, 0) {A};
              \draw (1, 0) circle (10pt);
              \node (B) at (1, 0) {B};
            \end{tikzpicture}
          }
        \end{center}
      \end{uncoverenv}
      \begin{uncoverenv}<4->
        \[ \fa{a \in A} \norm{p - a} > \alpha \]
        \vspace{-30pt}

        \begin{center}
          und
        \end{center}
        \vspace{-34pt}

        \[ \fa{b \in B} \norm{p - a} < \alpha \]
        \vspace{-20pt}
      \end{uncoverenv}
    \end{minipage}
  \end{definition}
\end{frame}

\begin{frame}
  \begin{theorem}[Kirchberger\only<2->{'}]
    % TODO: trennbar oder streng trennbar?
    Seien $P$ und $Q$ nichtleere, kompakte Teilmengen von $\E^n$. \\
    Dann sind $P$ und $Q$ genau dann durch eine \alt<2->{\alert<2>{Sphäre}}{Hyperebene} streng trennbar,
    wenn für jede Menge $T \subset \E^n$ mit maximal \alt<3->{\alert<3->{$n + 3$}}{$n + 2$} Punkten die Mengen $P \cap T$ und $Q \cap T$ durch eine \alt<2->{\alert<2>{Sphäre}}{Hyperebene}\\
    streng trennbar sind.
  \end{theorem}

  % Folgendes Beispiel im $\E^2$ zeigt, dass die Trennbarkeit von $n + 2$ Punkten nicht ausreicht:

  \begin{block}<4->{Beispiel im $\E^2$}
    \colorbox{white}{
      \begin{tikzpicture}
        \node (p1) [circle,fill,inner sep=1pt,label=180:$p_1$] at (0, 1) {};
        \node (p2) [circle,fill,inner sep=1pt,label=-90:$p_2$] at (0, -1) {};
        \node (p3) [circle,fill,inner sep=1pt,label=-45:$p_3$] at (2, 0) {};
        \draw [gray] (p1) -- (p2);
        \draw [gray] (p2) -- (p3);
        \draw [gray] (p3) -- (p1);
        \node<5-> (q1) [circle,fill,inner sep=1pt,label=225:$q_1$] at (0, 0) {};
        \node<6-> (q2) [circle,fill,inner sep=1pt,label=45:$q_2$] at (4, 0) {};
        \draw<7> [dashed,add= .5 and .5] (q1) to (q2);

      \end{tikzpicture}
    }
  \end{block}
\end{frame}

\section{Trennung durch Zylinder}

\begin{frame}
  TODO
\end{frame}

\section{Trennung durch Parallelotope}

\begin{frame}
  TODO
\end{frame}

\end{document}