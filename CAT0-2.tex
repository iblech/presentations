\documentclass{beamer}

\usepackage[utf8]{inputenc}
\usepackage[ngerman]{babel}

\usetheme{Frankfurt}
\useoutertheme[subsection=false]{miniframes}
\usepackage{color,graphicx,overpic,tikz}
\definecolor{DarkBlue}{rgb}{0.098,0.098,0.349} % RGB(25, 25, 89)
\setbeamercolor{section in head/foot}{fg=white, bg=DarkBlue}
\definecolor{LighterBlue}{rgb}{0.15,0.2,0.8}
\setbeamercolor{frametitle}{bg=LighterBlue, fg=white}
\setbeamercolor{title}{fg=white, bg=LighterBlue}
\setbeamercolor{structure}{fg=LighterBlue, bg=white}
\setbeamertemplate{headline}{}

\definecolor{LightBlue}{rgb}{0.5,0.65,1}

\newcommand{\E}{\mathbb{E}} % Euklidischer Raum
\newcommand{\R}{\mathbb{R}} % Reelle Zahlen
\newcommand{\I}{\left[0,1\right]} % Kompaktes Einheitsintervall
\newcommand{\HH}{\mathbb{H}} % Hyperbolischer Raum
\newcommand{\Bild}{\mathrm{Bild}} % Bild
\newcommand{\fa}[1]{\forall \, {#1} \,:\,} % Schöner Allquantor
\newcommand{\ex}[1]{\exists \, {#1} \,:\,} % Schöner Existenzquantor
\newcommand{\dist}{\mathsf{dist}} % Distanz

% Färbe \emph{} dunkelgrün
\definecolor{Emph}{rgb}{0.05,0.5,0.2}
\renewcommand{\emph}[1]{\textcolor{Emph}{#1}}

% Platzhalter
\newcommand{\blank}{\text{--}}

% Absolutwert, Norm
% siehe http://tex.stackexchange.com/questions/43008/absolute-value-symbols
\usepackage{mathtools}
\DeclarePairedDelimiter\norm{\lVert}{\rVert}
\DeclarePairedDelimiter\abs{\lvert}{\rvert}%

\theoremstyle{definition}

\newtheorem*{bspe}{Beispiele}
\newtheorem*{bsp}{Beispiel}
\newtheorem*{satz}{Satz}
\newtheorem*{lem}{Lemma}
\newtheorem*{bem}{Bemerkung}
\newtheorem*{kor}{Korollar}
\newtheorem*{prop}{Proposition}

% copied from http://tex.stackexchange.com/questions/11954/automatically-scale-big-and-small-graphics-for-beamer-presentations
\newcommand{\framedgraphic}[1] {
  \begin{frame}
    \begin{center}
      \vspace{-10pt}
      \includegraphics[width=\textwidth,height=0.85\textheight,keepaspectratio]{#1}
    \end{center}
  \end{frame}
}

\title{Alexandrov-Krümmung, Hadamard-Räume und der Satz von Cartan-Hadamard}
\author{Tim Baumann}
\institute{Seminar Metrische Geometrie}
\date{3. Juni 2014}

\begin{document}

\setlength{\abovedisplayskip}{2pt}
\setlength{\belowdisplayskip}{2pt}
\setlength{\abovedisplayshortskip}{2pt}
\setlength{\belowdisplayshortskip}{2pt}

\begin{frame}[plain]
  \titlepage
\end{frame}

\framedgraphic{bilder/Picture15.jpg}

\begin{frame}
  \frametitle{Wiederholung}

  Sei $(X, d)$ ein Längenraum.

  \begin{definition}
    Eine Teilmenge $U \subseteq X$ heißt \emph{CAT($0$)-Gebiet}, falls gilt:
    \begin{itemize}
      \item Für alle $x, y \in U$ gibt es eine Geodäte $\sigma_{xy} : \I \to U$ der Länge $d(x, y)$.
      \item Alle Dreiecke $\Delta abc$ mit Eckpunkten und Seiten in $U$ erfüllen die CAT($0$)-Vergleichseigenschaft:\\
      Für alle $d \in \Bild(\sigma_{ac})$ mit Vergleichspunkt $\overline{d}$ in $\Delta \overline{abc}$ gilt
      \[ d(b, d) \leq d_{\E^2}(\overline{b}, \overline{d}). \]
      und analog für $d' \in \sigma_{ab}$, $d'' \in \sigma_{bc}$.
    \end{itemize}
  \end{definition}

  \begin{definition}<2->
    Der Längenraum $X$ heißt \emph{CAT($0$)-Raum}, falls $X$ eine Überdeckung mit offenen CAT($0$)-Gebieten besitzt.\\
    Man sagt auch, der Raum habe \emph{Alexandrov-Krümmung $\leq 0$}.
  \end{definition}
\end{frame}

\begin{frame}
  \begin{lem}[BBI, 9.2.3]
    Sei $(X, d)$ ein Längenraum, $U \subseteq X$ ein CAT($0$)-Gebiet und $\alpha, \beta : I \to U$ zwei durch dasselbe Intervall $I$ parametrisierte und mit jeweils konstanter Geschwindigkeit durchlaufene Geodäten in $U$. Dann ist die Distanzfunktion
    \[ \delta : I \to \R_{\geq 0}, \quad t \mapsto d(\alpha(t), \beta(t)) \]
    konvex.
  \end{lem}
\end{frame}

\begin{frame}
  \begin{definition}
    Sei $X$ ein topologischer Raum, $\gamma_1, \gamma_2 : \left[ 0, 1 \right] \to X$ stetige Wege mit $p = \gamma_1(0) = \gamma_2(0)$ und $q = \gamma_1(1) = \gamma_2(1)$. Eine \emph{Homotopie} der Wege $\gamma_1$ und $\gamma_2$ relativ der Endpunkte ist eine stetige Abbildung
    \[ H : \I \times \I \to X \]
    mit
    \begin{itemize}
      \item $H(\blank, 0) = \gamma_1$,
      \item $H(\blank, 1) = \gamma_2$,
      \item $H(0, t) = p$ für alle $t \in \I$,
      \item $H(1, t) = q$ für alle $t \in \I$.
    \end{itemize}
  \end{definition}

  \begin{definition}<2->
    Ein topologischer Raum $X$ heißt \emph{einfach zusammenhängend}, falls
    \begin{itemize}
      \item er wegzusammenhängend ist und
      \item jeder geschlossene Weg $\gamma : \I \to X$ (d.\,h. $\gamma(0) = \gamma(1)$) homotop relativ der Endpunkte zum konstanten Weg $t \mapsto \gamma(0)$ ist.
    \end{itemize}
  \end{definition}
\end{frame}

\begin{frame}
  \begin{definition}
    
  % TODO: Surjektivität fordern oder folgern?
    Eine \emph{Überlagerung} eines topologischen Raumes $X$ ist ein Tupel $(\tilde{X}, \pi)$ bestehend aus einem topolischen Raum $\tilde{X}$ und einer surjektiven Überlagerungsabbildung $\pi : \tilde{X} \to X$, für die gilt:

    Für alle $x \in X$ gibt es eine Umgebung $U_x \subseteq X$, sodass
    \[ \pi^{-1}(U_x) = \bigcup_{i \in I} V_i \]
    für disjunkte offene Mengen $(V_i)_{i \in I}$, für die jeweils gilt:
    \[ \pi|_{V_i} : V_i \to U \]
    ist ein Homöomorphismus.
  \end{definition}

  \begin{bsp}<2->
    Jeder Homöomorphismus ist auch eine Überlagerung.
  \end{bsp}
\end{frame}

% Standardbeispiel: Überlagerung $\R \to S^1$ texen? (tikZ?)

\begin{frame}
  Überlagerungsabbildungen $\pi : \tilde{X} \to X$ besitzen folgende Hochhebungseigenschaften:

  % Faserschreibweise verwenden?
  \begin{lem}[Hochheben von Wegen]
    Sei $\gamma : \I \to X$ ein stetiger Weg und $z \in \tilde{X}$ mit $\pi(z) = \gamma(0)$.
    % (kurz: $z \in p^{-1}(\gamma(0))$).
    Dann gibt es genau einen Weg $\tilde{\gamma} : \I \to \tilde{X}$ mit
    \[
      \tilde{\gamma}(0) = z
      \quad \text{und} \quad
      p \circ \tilde{\gamma} = \gamma.
    \]
  \end{lem}

  \begin{lem}<2->[Hochheben von Weghomotopien]
    Seien $\gamma_1, \gamma_2 : \I \to X$ zwei stetige Wege mit $p := \gamma_1(0) = \gamma_2(0)$ und $\gamma_1(1) = \gamma_2(1)$ zusammen mit einer Homotopie $H : \I \times \I \to X$ zwischen $\gamma_1$ und $\gamma_2$ relativ der Endpunkte. Sei $z \in \tilde{X}$ mit $\pi(z) = p$ und $\tilde{\gamma_1}, \tilde{\gamma_2} : \I \to \tilde{X}$ die Hochhebungen von $\gamma_1$ bzw. $\gamma_2$ wie in obigem Lemma. Dann gibt es genau eine Homotopie
    \[ \tilde{H} : \I \times \I \to \tilde{X} \]
    von $\tilde{\gamma_1}$ und $\tilde{\gamma_2}$ relativ der Endpunkte.
  \end{lem}
\end{frame}

\begin{frame}
  \begin{definition}
    Ein \emph{lokaler Homöomorphismus} ist eine stetige Abbildung $f : Y \to Z$ zwischen topologischen Räumen, für die gilt:

    Für alle $y \in Y$ eine Umgebung $U_y \subseteq Y$ von $y$ existiert, sodass $f(U_y)$ offen und
    \[ f|_{U_y} : U_y \to f(U_y) \]
    ein Homöomorphismus ist.
  \end{definition}

  \begin{bem}<2->
    Jeder Überlagerungsabbildung $p : \tilde{X} \to X$ ist ein lokaler Homöomorphismus, aber nicht jeder lokale Homöomorphismus ist eine Überlagerungsabbildung.
  \end{bem}
\end{frame}

\begin{frame}
  \begin{definition}
    Ein vollständiger, einfach zusammenhängender Längenraum mit Alexandrov-Krümmung $\leq 0$ (kurz: ein CAT($0$)-Raum) heißt \emph{Hadamard-Raum}.
  \end{definition}
\end{frame}

\begin{frame}
  % BH = Martin Bridson, André Haefliger
  \begin{lem}[BH, 4.3]
    Sei $(X, d)$ ein vollständiger CAT($0$)-Raum und $c : \I \to X$ eine Geodäte von $x \coloneqq c(0)$ nach $y \coloneqq c(1)$. Sei $\epsilon > 0$ so klein, dass $\overline{B_{2 \epsilon}(c(t))}$ für alle $t \in \I$ eine CAT($0$)-Umgebung ist. Dann gilt:
    \begin{enumerate}
      \item Für alle $\overline{x} \in B_\epsilon(x)$ und $\overline{y} \in B_\epsilon(y)$ gibt es genau eine Geodäte $\overline{c} : \I \to X$ von $\overline{x}$ nach $\overline{y}$, sodass
      \[ t \mapsto d(c(t), \overline{c}(t)) \]
      konvex ist.
      \item Außerdem gilt: $L(\overline{c}) \leq d(x, \overline{x}) + L(c) + d(\overline{y}, y)$
    \end{enumerate}
  \end{lem}

  \begin{bem}<2->
    Solch ein $\epsilon > 0$ existiert aufgrund der Kompaktheit von $c(\I)$.
  \end{bem}
\end{frame}

\begin{frame}
  \begin{definition}
    Sei $X$ ein metrischer Raum und $p \in X$. Dann wird
    \begin{align*}
      \tilde{X}_p \coloneqq \{ &\text{Geodäten } \gamma : \I \to X \text{ mit } \gamma(0) = p\\
      &\text{und $\gamma$ mit konstanter Geschwindigkeit parametrisiert} \}
    \end{align*}
    \emph{Raum der Geodäten mit Startpunkt $p$} genannt. Mit der Metrik
    \[ d(\alpha, \beta) \coloneqq \max_{t \in \I} \abs{\alpha(t) - \beta(t)} \]
    wird $(\tilde{X}_p, d)$ zu einem metrischen Raum.\\
    Der Punkt $\overline{p} \in \tilde{X}_p$ sei die Konstante Geodäte $t \mapsto p$.
  \end{definition}

  \begin{definition}<2->
    Die \emph{Exponentialabbildung} ist die Abbildung
    \[ \exp_p : \tilde{X}_p \to X, \quad \gamma \mapsto \gamma(1), \]
    welche jede Geodäte auf ihren Endpunkt abbildet.
  \end{definition}
\end{frame}

\begin{frame}
  % Wie sehr ist dieses Lemma notwendig?
  \begin{lem}[BH, 4.5]
    Sei $X$ ein vollständiger CAT($0$)-Raum. Dann gilt:
    \begin{enumerate}
      \item $\tilde{X}_p$ ist zusammenziehbar.
      \item $\exp_p$ ist eine lokale Isometrie. % Was ist damit genau gemeint?
      \item Für alle $c {\in} \tilde{X}_p$ gibt es genau eine Geodäte in $\tilde{X}_p$ von $\tilde{p}$ nach $c$.
    \end{enumerate}
  \end{lem}
\end{frame}

\begin{frame}
  \begin{lem}[BH, 4.6]
    Sei $X$ ein vollständiger CAT($0$)-Raum und $p \in X$. Dann ist auch $\tilde{X}_p$ vollständig.
  \end{lem}
\end{frame}

% Induzierte Längenstrukturen. Notwendig?
% Lemma BH, 4.7.
% Hilfslemma (1.108 bzw. BH, I.3.28)
% Letztendliche Formulierung des Satzes von Cartan-Hadamard?
% Eindeutigkeit von Geodäten
% "Alexandrov's patchwork": CAT(0)-Vergleichseigenschaft auch im Großen.
% Dazu: Alexandrovs Lemma, "Verklebelemma"

\end{document}