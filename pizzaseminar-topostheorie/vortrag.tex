\documentclass{article}

\usepackage[utf8]{inputenc}
\usepackage[ngerman]{babel}
\usepackage[a4paper]{geometry}

\geometry{margin=2cm}
\setlength{\parindent}{0pt}

\usepackage{amsmath,amsthm,amssymb}
\usepackage{stmaryrd} % \mapsfrom
\usepackage{enumitem}

\usepackage{tikz}
%\usetikzlibrary{matrix,arrows,cd}
\usetikzlibrary{matrix}

\theoremstyle{definition}
\newtheorem*{defn}{Definition}
\newtheorem*{satz}{Satz}
\newtheorem*{satzdefn}{Satz/Definition}
\newtheorem*{lem}{Lemma}
\newtheorem*{bsp}{Beispiel}
\newtheorem*{bspe}{Beispiele}
\newtheorem*{kor}{Korollar}
\newtheorem*{kordefn}{Korollar/Definition}

\theoremstyle{remark}
\newtheorem*{erinnerung}{Erinnerung}
\newtheorem*{bem}{Bemerkung}
\newtheorem*{interp}{Interpretation}

% Miscellanea
\newcommand{\blank}{\text{--}} % Platzhalter
\newcommand{\coloneqq}{:=} % Definitionsgleich
% Schöne Mengen { #1 | #2 }
% siehe http://tex.stackexchange.com/questions/13634/define-pretty-sets-in-latex-esp-how-to-do-the-condition-separator
\usepackage{mathtools}
\DeclarePairedDelimiterX\Set[2]{\lbrace}{\rbrace}%
 { #1 \,\delimsize|\, #2 }
\newcommand{\N}{\mathbb{N}} % Natürliche Zahlen

% Konzepte
\DeclareMathOperator{\Ob}{Ob} % Objekte (einer Kategorie)
%\DeclareMathOperator{\Mor}{Mor} % Morphismenmenge / -klasse
\DeclareMathOperator{\End}{End} % Endomorphismenmenge / -klasse
\DeclareMathOperator{\Hom}{Hom} % Homomorphisms
\DeclareMathOperator{\id}{id} % Identität
%\DeclareMathOperator{\Nat}{Nat} % Natürliche Transformationen
%\DeclareMathOperator{\dom}{dom} % Domain
\DeclareMathOperator{\codom}{codom} % Codomain
\newcommand{\op}{\mathrm{op}} % opposite category
%\DeclareMathOperator{\Aut}{Aut} % Automorphismengruppe
\newcommand{\ladj}{\dashv} % Links-adjungiert (left-adjoint)
%\newcommand{\Lim}{\lim} % Limes
%\DeclareMathOperator{\colim}{colim} % Kolimes
%\newcommand{\Colim}{\colim} % Kolimes
%\newcommand{\myint}[2]{{\textstyle \int\limits_{#1}^{#2}}}
%\newcommand{\EndC}[2]{\myint{#1}{} #2} % Ende
%\newcommand{\CoEndC}[2]{\myint{}{#1} #2} % Koende
%\DeclareMathOperator{\Ran}{Ran} % Rechts-Kan-Erweiterung
%\DeclareMathOperator{\Lan}{Lan} % Links-Kan-Erweiterung
%\newcommand{\IHom}{\underline{\Hom}} % Inner Homomorphisms

% Topostheorie
\DeclareMathOperator{\Sub}{Sub} % Partialordnung der Subobjekte
\DeclareMathOperator{\true}{true} % universeller Monomorphismus
\newcommand{\clos}[1]{\overline{{#1}}} % "Abschluss" (im Lawvere-Tierney-Sinne)
\newcommand{\sheafification}{\mathbf{a}} % sheafification, Garbifizierung

% Populäre Kategorien
\newcommand{\SetC}{\mathbf{Set}} % Kategorie der Mengen
\newcommand{\FinSetC}{\mathbf{FinSet}} % Kategorie der endlichen Mengen
%\newcommand{\kAlg}{k\text{-}\Alg} % Kategorie der k-Algebren
\newcommand{\Sh}{\mathbf{Sh}} % Kategorie der Garben
\newcommand{\FuncC}[2]{[{#1}, {#2}]} % Funktorkategorie

% Bezeichnungen für Variablen, die für Kategorien stehen
%\newcommand{\Aat}{\mathcal{A}} % Category-A
%\newcommand{\Bat}{\mathcal{B}} % Category-B
\newcommand{\Cat}{\mathcal{C}} % Category-C
\newcommand{\Dat}{\mathcal{D}} % Category-D
\newcommand{\Eat}{\mathcal{E}} % Category-E
%\newcommand{\Fat}{\mathcal{F}} % Category-F
%\newcommand{\Gat}{\mathcal{G}} % Category-G
%\newcommand{\HatC}{\mathcal{H}} % Category-H
%\newcommand{\Iat}{\mathcal{I}} % Category-I (Indexkategorie)
%\newcommand{\Jat}{\mathcal{J}} % Category-J (Indexkategorie)
%\newcommand{\Lat}{\mathcal{L}} % Category-L
%\newcommand{\MatC}{\mathcal{M}} % Category-M
%\newcommand{\NatC}{\mathcal{N}} % Category-N
%\newcommand{\Sit}{\mathcal{S}} % Situs-S
%\newcommand{\Sat}{\mathcal{S}} % Category-S
%\newcommand{\Wat}{\mathcal{W}} % Category-W (whaaaatt?)
%\newcommand{\Xat}{\mathcal{X}} % Category-X
%\newcommand{\Yat}{\mathcal{Y}} % Category-Y

% Makros für Adjunktionen. Geklaut von
% http://sma.epfl.ch/~werndli/latex/adjunction.pdf
% http://sma.epfl.ch/~werndli/latex/adjunction.tex
\usepackage[all]{xy}
\newcommand{\adj}[1][]{\def\ArgI{#1}\adjRelayI}
\newcommand{\adjRelayI}[1][]{\def\ArgII{#1}\adjRelayII}
\newcommand{\adjRelayII}[3][2.2em]{
  \ensuremath{\SelectTips{lu}{10}\xymatrix@C=#1@1{
    {#2\,}
    \ar@<1ex>[r]^-{\ArgI}^-{}="1" &
    {\,#3}
    \ar@<1ex>[l]^-{\ArgII}^-{}="2"
    \ar@{}"1";"2"|(.3){\hbox{}}="7"
    \ar@{}"1";"2"|(.7){\hbox{}}="8"
    \ar@{|-} "8" ;"7"
  }}
}
\newcommand{\radj}[1][]{\def\ArgI{#1}\radjRelayI}
\newcommand{\radjRelayI}[1][]{\def\ArgII{#1}\radjRelayII}
\newcommand{\radjRelayII}[3][2.2em]{
  \ensuremath{\SelectTips{lu}{10}\xymatrix@C=#1@1{
  {#2\,}
  \ar@<-1ex>[r]_-{\ArgI}^-{}="1" &
  {\,#3}
  \ar@<-1ex>[l]_-{\ArgII}^-{}="2"
  \ar@{}"1";"2"|(.3){\hbox{}}="7"
  \ar@{}"1";"2"|(.7){\hbox{}}="8"
  \ar@{|-} "7" ;"8"
  }}
}

% http://tex.stackexchange.com/a/49196
\newcommand*\mycirc[1]{%
  \begin{tikzpicture}[baseline=(C.base)]
  \node[draw,circle,inner sep=1pt,minimum size=3ex](C) {#1};
  \end{tikzpicture}
}

\title{Geometrische Morphismen; Eigenschaften von Topoi und Konstruktionen mit Topoi}
\author{Tim Baumann}
\date{13. April 2017}

\begin{document}

\maketitle

\section{Geometrische Morphismen}

\begin{defn}
  Ein \emph{geometrischer Morphismus} $f : \Eat \to \Dat$ zwischen Topoi ist ein Paar
  \[ \radj[f_*][f^*]{\Eat}{\Dat} \]
  von adjungierten Funktoren, wobei $f^*$ linksexakt ist, d.\,h. $f^*$ bewahrt endliche Limiten. \\
  Dabei heißt $f^*$ \emph{Urbildfunktor} und $f_*$ \emph{Direktes-Bild-Funktor}.
\end{defn}

\begin{erinnerung}
  Außerdem bewahrt $f_*$ Limiten und $f^*$ Kolimiten, denn:
  \begin{center}
    \emph{
      Left-Adjoints Preserve Colimits (LAPC), \quad
      Right-Adjoints Preserve Limits (RAPL)
    }
  \end{center}
\end{erinnerung}

\begin{bem}
  Aus dem Yoneda-Lemma folgt: Bei einer Adjunktion $F \ladj G$ ist $F$ eindeutig (bis auf Isomorphie) durch~$G$ bestimmt (und umgekehrt).
  Die Daten eines geometrischen Morphismus sind also schon allein durch $f^*$ oder $f_*$ gegeben.
\end{bem}

\begin{defn}
  Ein \emph{Punkt} eines Topos~$\Eat$ ist ein geometrischer Morphismus $\SetC \to \Eat$.
\end{defn}

\begin{bem}
  Diese Definition ergibt Sinn, da für $\Eat = \Sh(X)$ geometrische Morphismen $\SetC = \Sh(\{ \heartsuit \}) \to \Sh(X)$ in 1-zu-1-Korrespondenz zu stetigen Abbildungen $\{ \heartsuit \} \to X$, also Punkten von~$X$, stehen.
\end{bem}

TODO: warum sind Punkte interessant?

\begin{bsp}
  Sei $\Eat$ ein kovollständiger Topos (z.\,B. ein Grothendiecktopos).
  Dann hat der \emph{Globale-Schnitte-Funktor}
  \[
    \Gamma : \Eat \to \SetC, \quad
    E \mapsto \Hom_\Eat(1, E)
  \]
  einen Linksadjungierten, nämlich
  \[
    \Delta : \SetC \to \Eat, \quad
    S \mapsto \coprod_{s \in S} 1.
  \]
  Für diesen gelten
  \begin{align*} 
    \Delta(\{ \heartsuit \}) & =
    \coprod_{s \in \{ \heartsuit \}} 1 \cong 1 \\
    \Delta(S \times T) & =
    \enspace \coprod_{\mathclap{(s, t) \in S \times T}} 1 \enspace \cong
    \left( \coprod_{s \in S} 1 \right) \times \left( \coprod_{t \in T} 1 \right) =
    \Delta(S) \times \Delta(T)
  \end{align*} 
  Außerdem kann man zeigen, dass $\Delta$ auch Differenzkerne und somit alle endlichen Limiten erhält.
  Folglich ist
  \[ \radj[\Gamma][\Delta]{\Eat}{\SetC} \]
  ein geometrischer Morphismus.

  Es gibt auch keinen anderen geometrischen Morphismus $f : \Eat \to \SetC$, denn für jeden solchen Morphismus und jedes Objekt $E \in \mathcal{E}$ gilt:
  \[
    f_* E \cong
    \Hom_\SetC(1, f_* E) \cong
    \Hom_\Eat(f^* 1, E) \cong
    \Hom_\Eat(1, E) \cong
    \Gamma(E).
  \]
\end{bsp}

\section{Stetige Abbildungen induzieren geometrische Morphismen}

TODO

\section{Scheibenkategorien von Topoi sind Topoi}

% siehe IV.7. The slice Category as a Topos

\begin{satz}
Sei $\Eat$ ein Topos, $B \in \Ob(\Eat)$.
Dann ist auch die Scheibenkategorie $\Eat / B$ ein Topos.
\end{satz}

\begin{proof}
Wir müssen die Toposaxiome nachrechnen:
\begin{enumerate}[itemsep=10pt,label=\protect\mycirc{\arabic*}]
\item $\Eat/B$ ist endlich vollständig:

\begin{minipage}[t]{0.35 \linewidth}
Der Differenzkern berechnet sich \\
wie in~$\Eat$:

\begin{tikzpicture}
  \matrix (mat) [matrix of nodes, column sep=1cm, row sep=1cm]{
    \node (K) {$K$}; &
    \node (X) {$X$}; &&
    \node (Y) {$Y$}; \\
    && \node (B) {$B$}; \\
  };
  \draw[->] (K) to node [above] {} (X);
  \draw[->, transform canvas={yshift=0.5ex}] (X) to node [above] {$f$} (Y);
  \draw[->, transform canvas={yshift=-0.5ex}] (X) to node [below] {$g$} (Y);
  \draw[->] (K) to node {} (B);
  \draw[->] (X) to node {} (B);
  \draw[->] (Y) to node {} (B);
\end{tikzpicture}
\end{minipage}
\begin{minipage}[t]{0.3 \linewidth}
Binäre Produkte in~$\Eat/B$ \\
sind Pullbacks in~$\Eat$:

\begin{tikzpicture}
  \matrix (mat) [matrix of nodes, column sep=1cm, row sep=0.6cm]{
    & \node (P) {$X \times_B Y$}; \\
    \node (X) {$X$}; &&
    \node (Y) {$Y$}; \\
    & \node (B) {$B$}; \\
  };
  \draw[->] (P) to node {} (Y);
  \draw[->] (P) to node {} (X);
  \draw[->] (P) to node {} (B);
  \draw[->] (X) to node {} (B);
  \draw[->] (Y) to node {} (B);
\end{tikzpicture}
\end{minipage}
\begin{minipage}[t]{0.35 \linewidth}
Das terminale Objekt in~$\Eat/B$ \\
ist der Morphismus $\id_B : B \to B$:

\begin{tikzpicture}
  \matrix (mat) [matrix of nodes, column sep=1.2cm, row sep=1cm]{
    \node (X) {$X$}; &&
    \node (B') {$B$}; \\
    & \node (B) {$B$}; \\
  };
  \draw[->] (X) to node [above] {$f$} (B');
  \draw[->] (X) to node [left] {$f$} (B);
  \draw[->] (B') to node [right] {$\id_B$} (B);
\end{tikzpicture}
\end{minipage}

\item $\Eat/B$ besitzt einen Unterobjektklassifizierer:

Sei $U : \Eat/B -> \Eat$ der offensichtliche Vergissfunktor.
Dieser Funktor ist linksadjungiert zum Funktor $(\blank \times B) : \Eat \to \Eat/B$, denn es gilt
\[ \Hom_\Eat(X, Y) \Hom_{\Eat/B}(X \xrightarrow{f} B, Y \times B \xrightarrow{\pi_2} B) \]
natürlich in $(X \xrightarrow{f} B) \in \Ob(\Eat/B)$ und $Y \in \Ob(\Eat)$.
Da ein Morphismus~$f$ in~$\Eat/B$ genau dann ein Monomorphismus ist, wenn $U(f)$ ein solcher ist, gilt:
\[
  \Sub_{\Eat/B}(A \xrightarrow{f} B) \cong
  \Sub_\Eat(A) \cong
  \Hom_\Eat(A, \Omega_\Eat) \cong 
  \Hom_{\Eat/B}(A \xrightarrow{f} B, \Omega_\Eat \times B \xrightarrow{\pi_2} B)
\]
Nach Proposition I.3.1 in SiGaL ist folglich $\Omega_{\Eat/B} = \Omega_\Eat \times B \xrightarrow{\pi_2} B$ der Unterobjektklassifizierer in~$\Eat/B$.
Der universelle Monomorphismus $\true : 1 \to \Omega$ ist die Komposition

\begin{center}\begin{tikzpicture}
  \matrix (mat) [matrix of nodes, column sep=2cm, row sep=1cm]{
    \node (B) {$B$}; &
    \node (B') {$1 \times B$}; &
    \node (Omega) {$\Omega \times B$}; \\
    & \node (B'') {$B$}; \\
  };
  \draw[->] (B) to node [above] {${\cong}$} (B');
  \draw[->] (B') to node [above] {$\true_\Eat \times \id_B$} (Omega);
  \draw[->] (B) to node [left] {$\id_B$} (B'');
  \draw[->] (B') to node [right] {$\pi_2$} (B'');
  \draw[->] (Omega) to node [right] {$\pi_2$} (B'');
\end{tikzpicture}\end{center}

\item $\Eat/B$ ist kartesisch abgeschlossen:

TODO: gibt es hierfür eine schöne Konstruktion mithilfe der internen Sprache?
Die Konstruktion in SiGaL ist zu kompliziert und nimmt außerdem $Z = \Omega$ an.
Vergleich mit Set: Dort gilt $Z^Y = \amalg_{b \in B} Z_b^{Y_b}$.

\end{enumerate}

\end{proof}

TODO: Evtl. erwähnen wie Kolimiten berechnet werden

\begin{defn}
  Sei $k : B \to A$ ein Morphismus in~$\Eat$.
  Dann erhalten wir einen Funktor
  \[
    \Sigma_k : \Eat/B \to \Eat/A, \quad
    (X \xrightarrow{p_X} B) \mapsto (k \circ p_X : X \xrightarrow{p_X} B \xrightarrow{k} A)
  \]
  durch Komponieren mit~$k$ und einen \emph{Basiswechselfunktor}
  \[
    k^* : \Eat/A \to \Eat/B, \quad
    (X \xrightarrow{p_X} A) \mapsto (B \times_A X \xrightarrow{\pi_B} B)
  \]
  durch Pullback entlang~$k$.
\end{defn}

\begin{lem}
  $\Sigma_k \ladj k^*$
\end{lem}

\begin{proof}
Wir müssen zeigen, dass
\[
  \Hom_{\Eat/A}(\Sigma_k(X \to B), Y \to A) \cong \Hom_{\Eat/B}(X \to B, k^*(Y \to A)).
\]
Betrachte das Diagramm

\begin{center}\begin{tikzpicture}
  \matrix (mat) [matrix of nodes, column sep=2cm, row sep=1cm]{
    \node (X) {$X$}; \\
    & \node (P) {$B \times_A Y$}; &
    \node (Y) {$Y$}; \\
    & \node (B) {$B$}; &
    \node (A) {$A$}; \\
  };
  \draw[->] (X) to node {} (B);
  \draw[->,dashed] (X) to node {} (P);
  \draw[->,dashed] (X) to node {} (Y);
  \draw[->] (P) to node {} (Y);
  \draw[->] (P) to node {} (B);
  \draw[->] (B) to node {$k$} (A);
  \draw[->] (Y) to node {} (A);
  %\draw[->] (B) to node [above] {${\cong}$} (B');
\end{tikzpicture}\end{center}

Elemente der linken Hom-Menge sind Morphismen $X \to Y$, die das äußere Viereck kommutieren lassen; Elemente der rechten Hom-Menge sind Morphismen $X \to B \times_A Y$, die das linke Dreieck kommutieren lassen.
Zwischen solchen Elementen besteht eine 1-zu-1-Korrespondenz, gegeben durch die universelle Eigenschaft des Pullbacks~$B \times_A Y$.
\end{proof}

\begin{lem}
  $k^*$ besitzt auch einen Rechtsadjungierten $\Pi_k : \Eat/B \to \Eat/A$.
\end{lem}

\begin{proof}
  Wir dürfen annehmen, dass $A \cong 1$ und somit $\Eat/A \cong \Eat$. (Ansonsten verwende $\Eat' \coloneqq \Eat/A$ und $B' \coloneqq (B \xrightarrow{k} A) \in \Ob(\Eat')$ anstelle von~$\Eat$ bzw.~$B$. Beachte, dass $\Eat'/B' \simeq \Eat/B$.)

  \begin{align*}
    \Hom_{\Eat/B}(k^*(X), Y \xrightarrow{h} B)
    & \cong \Hom_{\Eat/B}(X \times B \xrightarrow{\pi_B} B, Y \xrightarrow{h} B) \\
    & \cong \Set{t \in \Hom_\Eat(X \times B, Y)}{h \circ t = \pi_B} \\
    & \cong \Set{t' \in \Hom_\Eat(X, Y^B)}{h^B \circ t' = j \circ {!}} \\
    & \cong \Hom_\Eat(X, \Set{g : Y^B}{h \circ g = \id_B})
  \end{align*}
  wobei $j : 1 \to B^B$ die Curryfizierung von $\id_B$ und $! : X \to 1$ ist.
  Wir definieren somit $\Pi_k$ durch
  \[
    \Pi_k(k : Y \to B) \coloneqq \Set{g : Y^B}{h \circ g = \id_B}.
    \qedhere
  \]
\end{proof}

\begin{kor}
  $\radj[\Sigma_k][k^*]{\Eat/B}{\Eat/A}$
  ist ein geometrischer Morphismus. \\
  Dieser ist \emph{wesentlich}, d.\,h. $k^*$ besitzt auch einen Linksadjungierten.
\end{kor}

\section{Lawvere-Tierney-Topologien und Garbifizierung}

\begin{defn}
  Eine \emph{Lawvere-Tierney-Topologie} auf einem Topos~$\Eat$ ist ein Morphismus $j : \Omega \to \Omega$, für den gilt:
  \begin{center}
  \begin{minipage}[t]{0.2 \linewidth}
    (a) $j \circ \true = \true$

    \begin{tikzpicture}
      \matrix (mat) [matrix of nodes, column sep=1cm, row sep=1cm]{
        \node (One) {$1$}; &
        \node (Omega') {$\Omega$}; \\
        & \node (Omega'') {$\Omega$}; \\
      };
      \draw[->] (One) to node [above] {$\true$} (Omega');
      \draw[->] (One) to node [left] {$\true$} (Omega'');
      \draw[->] (Omega') to node [right] {$j$} (Omega'');
    \end{tikzpicture}
  \end{minipage}
  \begin{minipage}[t]{0.2 \linewidth}
    (b) $j \circ j = j$

    \begin{tikzpicture}
      \matrix (mat) [matrix of nodes, column sep=1cm, row sep=1cm]{
        \node (Omega) {$\Omega$}; &
        \node (Omega') {$\Omega$}; \\
        & \node (Omega'') {$\Omega$}; \\
      };
      \draw[->] (Omega) to node [above] {$j$} (Omega');
      \draw[->] (Omega) to node [left] {$j$} (Omega'');
      \draw[->] (Omega') to node [right] {$j$} (Omega'');
    \end{tikzpicture}
  \end{minipage}
  \begin{minipage}[t]{0.2 \linewidth}
    (b) $j \circ {\wedge} = \wedge \circ (j \times j)$

    \begin{tikzpicture}
      \matrix (mat) [matrix of nodes, column sep=1cm, row sep=1cm]{
        \node (Omega2) {$\Omega \times \Omega$}; &
        \node (Omega) {$\Omega$}; \\
        \node (Omega2') {$\Omega \times \Omega$}; &
        \node (Omega') {$\Omega$}; \\
      };
      \draw[->] (Omega2) to node [above] {$\wedge$} (Omega);
      \draw[->] (Omega2') to node [above] {$\wedge$} (Omega');
      \draw[->] (Omega2) to node [left] {$j \times j$} (Omega2');
      \draw[->] (Omega) to node [left] {$j$} (Omega');
    \end{tikzpicture}
  \end{minipage}
  \end{center}
\end{defn}

\begin{interp}
  $j$ ist ein idempotenter, mit ${\wedge}$ und $\true$ verträglicher modaler Operator

  Zum Beispiel: Sei $\varphi$ eine Aussage. Die anschauliche Bedeutung von $\square \varphi$ ist "`$\varphi$ gilt immer"'.
  Dann sollten intuitiv auch folgende Regeln gelten:
  \[
    \square \top = \top, \qquad
    \square \square \varphi \iff \square \varphi
    \qquad \text{sowie} \qquad
    (\square \varphi) \wedge (\square \psi) \iff \square (\varphi \wedge \psi)
  \]
  Solch ein Operator $\square$ sollte also eine Lawvere-Tierney-Topologie stiften.
  Im Gegensatz dazu stiftet der Operator $\lozenge$ mit der Interpretation "`$\lozenge \varphi$ gilt, falls $\varphi$ möglich ist"', denn aus $\lozenge \varphi \wedge \lozenge \psi$ folgt i.\,A. nicht $\lozenge (\varphi \wedge \psi)$.
\end{interp}

\begin{defn}
  Für ein Unterobjekt $A \hookrightarrow E$ ist $\clos{A} \hookrightarrow E$ dasjenige Unterobjekt mit
  \[ \chi_{\clos{A}} = j \circ \chi_A : E \to \Omega. \]
\end{defn}

\begin{lem}
  \begin{minipage}[t]{0.8 \linewidth}\begin{itemize}
    \item
      $A \subseteq \clos{A}$, \quad
      $\clos{\clos{A}} = \clos{A}$, \quad
      $\clos{A \cap B} = \clos{A} \cap \clos{B}$
    \item $f^{-1}(\clos{A}) = \clos{f^{-1}(A)}$ \enspace (Natürlichkeit) \qquad
  \end{itemize}\end{minipage}
\end{lem}

\begin{defn}
  Sei $j$ eine Lawvere-Tierney-Topologie auf~$\Eat$.
  \begin{itemize}[itemsep=0pt]
    \item Ein Unterobjekt $A \hookrightarrow E$ heißt \emph{dicht}, falls $\clos{A} = E$.
    \item
      Eine \emph{$j$-Garbe} ist ein Objekt~$F \in \Ob(\Eat)$, für das gilt: \\
      Für alle dichten Unterobjekte $A \xhookrightarrow{m} E$ ist $m^* : \Hom(E, F) \to \Hom(A, F)$ ein Isomorphismus.
    \item $\Sh_j(\Eat)$ ist die volle Unterkategorie der $j$-Garben von~$\Eat$.
  \end{itemize}
\end{defn}

% V.2.5
\begin{satz}
  $\Sh_j(\Eat)$ ist ein Topos.
\end{satz}

\begin{proof}[Beweisskizze]
  Zeige:
  \begin{itemize}
    \item Die Unterkategorie $\Sh_j(\Eat)$ ist abgeschlossen unter der Bildung von Limiten und Exponentialobjekten.
    \item
      Sei $\Omega_j$ der Differenzkern

      \begin{tikzpicture}
        \matrix (mat) [matrix of nodes, column sep=1cm, row sep=1cm]{
          \node (Omegaj) {$\Omega_j$}; &
          \node (Omega) {$\Omega$}; &
          \node (Omega') {$\Omega$}; \\
        };
        \draw[->] (Omegaj) to node {} (Omega);
        \draw[->, transform canvas={yshift=0.5ex}] (Omega) to node [above] {$j$} (Omega');
        \draw[->, transform canvas={yshift=-0.5ex}] (Omega) to node [below] {$\id$} (Omega');
      \end{tikzpicture}

      Nenne ein Subobjekt $A \hookrightarrow F$ \emph{abgeschlossen}, falls $\clos{A} = A$.
      Für eine $j$-Garbe~$F$ zeige dann, dass
      \[
        \Sub_{\Sh_j(\Eat)}(F) =
        \Set{A \in \Sub_\Eat(F)}{\text{$A$ abgeschlossen}} \cong
        \Hom_\Eat(F, \Omega_j)
      \]
      und dass $\Omega_j$ eine $F$-Garbe ist.
      Somit ist $\Omega_j$ der Unterobjektklassifizierer von~$\Sh_j(\Eat)$. \qedhere
  \end{itemize}
\end{proof}

Sei $i : \Sh_j(\Eat) \to \Eat$ der Einbettungsfunktor.

% V.3.1
\begin{satzdefn}
  \begin{minipage}[t]{0.99 \linewidth}
    \begin{itemize}
      \item $i$ hat einen Linksadjungierten, die \emph{$j$-Garbifizierung} $\sheafification : \Eat \to \Sh_j(\Eat)$.
      \item $\sheafification$ ist linksexakt.
    \end{itemize}
  \end{minipage}
\end{satzdefn}

\begin{kordefn}
  $\radj[i][\sheafification]{\Sh_j(\Eat)}{\Eat}$
  ist ein geometrischer Morphismus. \\
  Dieser ist eine \emph{geometrische Einbettung}, d.\,h. der Direktes-Bild-Funktor $i$ ist volltreu.
  Dies macht $\Sh_j(\Eat)$ zu einem \emph{Untertopos} von~$\Eat$.
\end{kordefn}

\begin{bem}
  Bis auf Kategorienäquivalenz ist jede geometrische Einbettung von dieser Form.
\end{bem}

\begin{bsp}
  $\FinSetC$ ist kein Untertopos von~$\SetC$ vermöge der Inklusion $i : \FinSetC \to \SetC$, denn es gibt keine endliche Menge $X$ mit
  \[ \Hom_\FinSetC(X, \{ \heartsuit, \diamondsuit \}) \cong \Hom_\SetC(\N, i(\{ \heartsuit, \diamondsuit \})). \]
  Somit besitzt $i$ keinen Linksadjungierten.
\end{bsp}

% V.1.2, V.4.1
\begin{satz}
  Sei $\Cat$ eine kleine Kategorie und $\Eat \coloneqq \FuncC{\Cat^\op}{\SetC}$.
  Dann gibt es eine 1-zu-1-Korrespondenz
  \[
    \begin{array}{r c l}
      \{ \text{ Grothendieck-Topologien auf~$\Cat$ } \} & \leftrightarrow & \{ \text{ Lawvere-Tierney-Topologien auf~$\Eat$ } \} \\
      J & \mapsto & j_J \coloneqq ((j_J)_C : S \mapsto \Set{g}{\codom(g) = C, \text{ $S$ überdeckt $g$}})_{C \in \Ob(\Cat)} \\
      J_j \coloneqq j^*(1 \xhookrightarrow{\true} \Omega) & \mapsfrom & j
    \end{array}
  \]
  (Erinnerung: $\Omega \in \Ob(\Eat)$ ist die Prägarbe mit $\Omega(C) \coloneqq \{ \text{ Siebe auf~$C$ } \}$.)
\end{satz}

% V.4.2
\begin{satz}
  Desweiteren gilt für eine Prägarbe $P \in \Ob(\Eat)$:
  \[
    \text{$P$ ist $j$-Garbe}
    \iff
    \text{$P$ ist Garbe bzgl. $J_j$, also $P \in \Ob(\Sh(\Eat, J_j))$}.
  \]
\end{satz}

\begin{kor}
  Die Garbifizierungen einer Prägarbe bzgl. der Lawvere-Tierney-Topologie~$j$ oder der zugehörigen Grothendieck-Topologie~$J_j$ sind isomorph.
\end{kor}

% Achtung: Der Begriff einer Grothendieck-Topologie wurde im Seminar noch nicht definiert.
% Lukas hat am Anfang des zweiten Teils seines Vortrags den Begriff einer *Basis einer* Grothendieck-Topologie definiert.
% Dieser verhält sich zu Grothendieck-Topologien wie "Basis eines topol. Raums" zu "Topologie (eines topol. Raums)".

\section{Weitere Quellen für geometrische Morphismen}

% VII.2.2
\begin{satz}
  Sei $\phi : \Cat \to \Dat$ ein Funktor.
  Dann gibt es einen geometrischen Morphismus \\
  $\radj[\phi_*][\phi^*]{\FuncC{\Cat^\op}{\SetC}}{\FuncC{\Dat^\op}{\SetC}}$
  mit $\phi^*(P) \coloneqq (\Cat^\op \xrightarrow{\phi^\op} \Dat^\op \xrightarrow{P} \SetC)$.
  Dieser ist wesentlich. % d.\,h. $\phi^*$ besitzt auch einen Linksadjungierten
\end{satz}

\end{document}

TODO: Grothendieck-Topologie zu neg-neg (?)
TODO: Untertopoi von Grothendiecktopoi sind Grothendiecktopoi (?)
TODO: Faktorisierung von geometrischen Morphismen (?)
TODO: Punkte von Grothendieck-Topoi (?)

Geometrische Morphismen:
* logische Interpretation: "Ein geom. Mor. F \to E macht F zu einem E-Topos; aus interner Sicht von E wird diese Situation zu F -> Set"
* Eigenschaften von geometrischen Morphismen
  * offen/abgeschlossen
  * Surjektion

Beispiele für geometrische Morphismen:
* f : X -> Y  ~>  E/X -> E/Y (sogar wesentlich)
  * insbesondere Speziallfall E/Y = E/1 = E
* f : X -> Y stetig  ~>  (f^* \ladj f_*) : Sh(X) -> Sh(Y)
  * für nüchterne Räume X und Y: 1-zu-1-Korrespondenz
* F : E -> D Funktor  ~>  Psh(X) -> Psh(Y) (auch wesentlich)
* j : \Omega -> \Omega Lawvere-Tierney-Topologie auf E  ~>  E_j -> E
  * logische Interpretation
  * j = \neg \neg
  * j = (\varphi => _)
  * j = (_ \vee \varphi)

* TODO: Logische Morphismen (?)

Fragen:
* haben wesentliche geometrische Morphismen erwähnenswerte Eigenschaften