\documentclass{beamer}

\usepackage[utf8]{inputenc}
\usepackage[ngerman]{babel}

\usetheme{Frankfurt}
\useoutertheme[subsection=false]{miniframes}
\usepackage{color,graphicx,overpic,tikz}
\usetikzlibrary{matrix,shapes,arrows,positioning}
\definecolor{DarkBlue}{rgb}{0.098,0.098,0.349} % RGB(25, 25, 89)
\setbeamercolor{section in head/foot}{fg=white, bg=DarkBlue}
\definecolor{LighterBlue}{rgb}{0.15,0.2,0.8}
\setbeamercolor{frametitle}{bg=LighterBlue, fg=white}
\setbeamercolor{title}{fg=white, bg=LighterBlue}
\setbeamercolor{structure}{fg=LighterBlue, bg=white}
\setbeamercovered{transparent}
\setbeamertemplate{headline}{}

\definecolor{LightBlue}{rgb}{0.5,0.65,1}

\newcommand{\E}{\mathbb{E}} % Euklidischer Raum
\newcommand{\R}{\mathbb{R}} % Reelle Zahlen
\newcommand{\N}{\mathbb{N}} % Natürliche Zahlen
\newcommand{\I}{\left[0,1\right]} % Kompaktes Einheitsintervall
\newcommand{\HH}{\mathbb{H}} % Hyperbolischer Raum
\newcommand{\Bild}{\mathrm{Bild}} % Bild
\newcommand{\fa}[1]{\forall \, {#1} \,:\,} % Schöner Allquantor
\newcommand{\ex}[1]{\exists \, {#1} \,:\,} % Schöner Existenzquantor
\newcommand{\dist}{\mathsf{dist}} % Distanz

% Färbe \emph{} dunkelgrün
\definecolor{Emph}{rgb}{0.05,0.5,0.2}
\renewcommand{\emph}[1]{\textcolor{Emph}{#1}}

% Platzhalter
\newcommand{\blank}{\text{--}}

% Absolutwert, Norm
% siehe http://tex.stackexchange.com/questions/43008/absolute-value-symbols
\usepackage{mathtools}
\DeclarePairedDelimiter\norm{\lVert}{\rVert}
\DeclarePairedDelimiter\abs{\lvert}{\rvert}%

\theoremstyle{definition}

\newtheorem*{bspe}{Beispiele}
\newtheorem*{bsp}{Beispiel}
\newtheorem*{satz}{Satz}
\newtheorem*{lem}{Lemma}
\newtheorem*{bem}{Bemerkung}
\newtheorem*{kor}{Korollar}
\newtheorem*{prop}{Proposition}
\newtheorem*{folg}{Folgerung}
\newtheorem*{bew}{Beweis}

% copied from http://tex.stackexchange.com/questions/11954/automatically-scale-big-and-small-graphics-for-beamer-presentations
\newcommand{\framedgraphic}[1] {
  \begin{frame}
    \begin{center}
      \vspace{-10pt}
      \includegraphics[width=\textwidth,height=0.85\textheight,keepaspectratio]{#1}
    \end{center}
  \end{frame}
}

\title{Alexandrov-Krümmung, Hadamard-Räume und der Satz von Cartan-Hadamard}
\author{Tim Baumann}
\institute{Seminar Metrische Geometrie}
\date{17. Juni 2014}

\begin{document}

\setlength{\abovedisplayskip}{2pt}
\setlength{\belowdisplayskip}{2pt}
\setlength{\abovedisplayshortskip}{2pt}
\setlength{\belowdisplayshortskip}{2pt}

\begin{frame}[plain]
  \titlepage
\end{frame}

\iffalse
\begin{frame}
  \begin{prop}[BBI, 9.1.17]
    Sei $(X, d)$ ein Längenraum, $U \subseteq X$ ein CAT($0$)-Gebiet. Dann gilt:
    \begin{enumerate}
      \item Seien $\sigma_{ab}$ und $\sigma_{bc}$ zwei kürzeste Wege in $U$, die in $b$ enden bzw. starten. Falls $\measuredangle abc = \pi$, dann ist auch $\sigma_{ab} * \sigma_{bc}$ ein kürzester Weg.
      \item Jede Geodäte in $U$ ist ein kürzester Weg.
    \end{enumerate}
  \end{prop}
\end{frame}

\begin{frame}
  \begin{definition}
    Sei $(X, d)$ ein Längenraum. Eine Geodäte $\gamma : \left[a, b\right] \to X$ heißt \emph{linear parametrisiert}, wenn für alle $a \leq s < t \leq b$ gilt:
    \[ \frac{L(\gamma|_{\left[s,t\right]})}{L(\gamma)} = \frac{t-s}{b-a} \]
  \end{definition}

  \begin{lem}<2->[BBI, 9.2.3]
    Sei $(X, d)$ ein Längenraum, $U \subseteq X$ ein CAT($0$)-Gebiet und $\alpha, \beta : I \to U$ zwei durch dasselbe Intervall $I$ und linear parametrisierte Geodäten in $U$. Dann ist die Distanzfunktion
    \[ \delta : I \to \R_{\geq 0}, \quad t \mapsto d(\alpha(t), \beta(t)) \]
    konvex.
  \end{lem}

  \begin{kor}<3->[Eindeutigkeit]
    Sei oben $I = \left[t_1, t_2\right]$ und $\alpha(t_1) = \beta(t_1)$, $\alpha(t_2) = \beta(t_2)$. Dann gilt $\alpha \equiv \beta$ auf ganz $I$.
  \end{kor}
\end{frame}

\begin{frame}
  \begin{definition}
    Sei $X$ ein topologischer Raum, $\gamma_1, \gamma_2 : \left[ 0, 1 \right] \to X$ stetige Wege mit $p = \gamma_1(0) = \gamma_2(0)$ und $q = \gamma_1(1) = \gamma_2(1)$. Eine \emph{Homotopie} der Wege $\gamma_1$ und $\gamma_2$ \emph{relativ der Endpunkte} ist eine stetige Abbildung
    \[ H : \I \times \I \to X \]
    mit
    \begin{itemize}
      \item $H(\blank, 0) = \gamma_1$,
      \item $H(\blank, 1) = \gamma_2$,
      \item $H(0, t) = p$ für alle $t \in \I$,
      \item $H(1, t) = q$ für alle $t \in \I$.
    \end{itemize}
  \end{definition}

  \begin{definition}<2->
    Ein topologischer Raum $X$ heißt \emph{einfach zusammenhängend}, falls
    \begin{itemize}
      \item er wegzusammenhängend ist und
      \item jeder geschlossene Weg $\gamma : \I \to X$ (d.\,h. $\gamma(0) = \gamma(1)$) homotop relativ der Endpunkte zum konstanten Weg $t \mapsto \gamma(0)$ ist.
    \end{itemize}
  \end{definition}
\end{frame}

\begin{frame}
  \begin{definition}
    Ein vollständiger, einfach zusammenhängender Längenraum mit Alexandrov-Krümmung $\leq 0$ heißt \emph{Hadamard-Raum}.
  \end{definition}

  \begin{satz}[Cartan-Hadamard]
    \begin{enumerate}
      \item Für alle Paare $p, q$ von Punkten in einem Hadamard-Raum gibt es genau eine verbindende Geodäte $\sigma_{pq}$.
      \item All diese Geodäten sind kürzeste Wege.
    \end{enumerate}
  \end{satz}
\end{frame}

\begin{frame}
  \begin{lem}[Konvergenz von linear param. Geodäten]
    Sei $X$ ein vollständiger CAT($0$)-Raum und $(\gamma_n : \I \to X)_{n \in \N}$ eine Cauchy-Folge bestehend linear parametrisierten Geodäten bzgl. der Maximumsmetrik
    \[ d(\alpha, \beta) \coloneqq \max_{t \in \I} d(\alpha(t), \beta(t)). \]
    Dann ist die Grenzfunktion
    \[ \gamma : \I \to X, \quad t \mapsto \lim_{n \to \infty} \gamma_n(t) \]
    eine linear parametrisierte Geodäte.
  \end{lem}
\end{frame}
\fi

\begin{frame}
  % BH = Martin Bridson, André Haefliger
  \begin{lem}[BH, II.4.3]
    Sei $(X, d)$ ein vollständiger CAT($0$)-Raum und $c : \I \to X$ eine Geodäte von $x \coloneqq c(0)$ nach $y \coloneqq c(1)$. Sei $\epsilon > 0$ so klein, dass $\overline{B_{2 \epsilon}(c(t))}$ für alle $t \in \I$ ein CAT($0$)-Gebiet ist. Dann gilt:
    \begin{enumerate}
      \item Seien $\alpha, \beta : \I \to X$ linear parametrisierte Geodäten mit
      \[
        d(\alpha(t), c(t)) < \epsilon
        \quad \text{und} \quad
        d(\beta(t), c(t)) < \epsilon
        \quad \text{$\forall t \in \I$}.
      \]
      Dann ist die Abstandsfunktion $\delta(t) \coloneqq d(\alpha(t), \beta(t))$ konvex.
      \item \uncover{Für alle $\overline{x} \in B_\epsilon(x)$ und $\overline{y} \in B_\epsilon(y)$ gibt es genau eine Geodäte $\overline{c} : \I \to X$ von $\overline{x}$ nach $\overline{y}$, sodass
      \[ t \mapsto d(c(t), \overline{c}(t)) \]
      konvex ist.}
      \item \uncover<2->{Außerdem gilt: $L(\overline{c}) \leq d(x, \overline{x}) + L(c) + d(\overline{y}, y)$}
    \end{enumerate}
  \end{lem}

  \begin{bem}
    Solch ein $\epsilon > 0$ existiert aufgrund der Kompaktheit von $c(\I)$.
  \end{bem}
\end{frame}


\begin{frame}
  \begin{definition}
    Sei $X$ ein metrischer Raum und $p \in X$. Dann wird
    \begin{align*}
      \tilde{X}_p \coloneqq \{ &\text{Geodäten } \gamma : \I \to X \text{ mit } \gamma(0) = p\\
      &\text{und $\gamma$ linear parametrisiert} \}
    \end{align*}
    \emph{Raum der Geodäten mit Startpunkt $p$} genannt. Mit der Metrik
    \[ d(\alpha, \beta) \coloneqq \max_{t \in \I} \abs{\alpha(t) - \beta(t)} \]
    wird $(\tilde{X}_p, d)$ zu einem metrischen Raum.\\
    Der Punkt $\tilde{p} \in \tilde{X}_p$ sei die konstante Geodäte $t \mapsto p$.
  \end{definition}

  \begin{definition}<2->
    Die \emph{Exponentialabbildung} ist die Abbildung
    \[ \exp_p : \tilde{X}_p \to X, \quad \gamma \mapsto \gamma(1), \]
    welche jede Geodäte auf ihren Endpunkt abbildet.
  \end{definition}
\end{frame}

\begin{frame}
  \begin{lem}<1->[BH, II.4.6]
    Sei $X$ ein vollständiger CAT($0$)-Raum und $p \in X$. Dann ist auch $\tilde{X}_p$ vollständig.
  \end{lem}

  \begin{lem}<2->[BH, II.4.5]
    Sei $X$ ein vollständiger CAT($0$)-Raum und $p \in X$. Dann gilt:
    \begin{enumerate}
      \item $\tilde{X}_p$ ist zusammenziehbar.
      \item Für alle $\tilde{x} \in \tilde{X}_p$ existiert ein $r > 0$, sodass $\exp_p(B_r(\tilde{x})) = B_r(\exp_p(\tilde{x}))$ und
      \[ \exp_p|_{B_r(\tilde{x})} : B_r(\tilde{x}) \to B_r(\exp_p(\tilde{x})) \]
      eine Isometrie ist. Insbesondere ist $\exp_p : \tilde{X}_p \to X$ ist eine lokale Isometrie.
    \end{enumerate}
  \end{lem}

  \begin{kor}<3->
    $\tilde{X}_p$ ist einfach zusammenhängend.
  \end{kor}
\end{frame}

\begin{frame}
  \begin{definition}
    Seien $X$, $Y$ topologische Räume und $p : X \to Y$ stetig. Eine Teilmenge $U \subset Y$ wird von $p$ \emph{gleichmäßig überlagert}, falls es einen diskreten topologischen Raum $D$ und einen Homöomorphismus $\phi : p^{-1}(U) \to U \times D$ gibt, sodass kommutiert:
    \begin{center}
      \begin{tikzpicture}
        \matrix (mat) [matrix of nodes, column sep=0.8cm, row sep=0.95cm, ampersand replacement=\&]{
          \node (PM) {$p^{-1}(U)$}; \& \&
          \node (UD) {$U \times D$}; \\
          \& \node (U) {$U$}; \\
        };
        \draw[->] (PM) to node [above] {$\phi$} (UD);
        \draw[->] (PM) to node [below] {$\approx$} (UD);
        \draw[->] (UD) to node [above] {$\pi$} (U);
        \draw[->] (PM) to node [left] {$p$} (U);
      \end{tikzpicture}
    \end{center}

    Die Abbildung $p$ ist eine \emph{Überlagerung}, falls jeder Punkt in $y$ eine gleichmäßig überlagerte Umgebung besitzt.
  \end{definition}

  \begin{bsp}<2->
    Jeder Homöomorphismus ist auch eine Überlagerung.
  \end{bsp}
\end{frame}

% Standardbeispiel: Überlagerung $\R \to S^1$ texen? (tikZ?)

\begin{frame}
  Überlagerungsabbildungen $p : \tilde{X} \to X$ besitzen folgende wichtige Hochhebungseigenschaften:

  % Faserschreibweise verwenden?
  \begin{lem}[Hochheben von Wegen]
    Sei $\gamma : \I \to X$ ein stetiger Weg und $\tilde{x}_0 \in \tilde{X}$ mit $p(\tilde{x}_0) = \gamma(0)$.
    % (kurz: $\tilde{x}_0 \in p^{-1}(\gamma(0))$).
    Dann gibt es genau einen Weg $\tilde{\gamma} : \I \to \tilde{X}$ mit
    \[
      \tilde{\gamma}(0) = \tilde{x}_0
      \quad \text{und} \quad
      p \circ \tilde{\gamma} = \gamma.
    \]
  \end{lem}

  \begin{lem}<2->[Hochheben von Weghomotopien]
    Seien $\gamma_1, \gamma_2 : \I \to X$ zwei stetige Wege mit $x_0 := \gamma_1(0) = \gamma_2(0)$ und $\gamma_1(1) = \gamma_2(1)$ zusammen mit einer Homotopie $H : \I \times \I \to X$ zwischen $\gamma_1$ und $\gamma_2$ relativ der Endpunkte. Sei $\tilde{x}_0 \in \tilde{X}$ mit $p(\tilde{x}_0) = x_0$ und $\tilde{\gamma_1}, \tilde{\gamma_2} : \I \to \tilde{X}$ die Hochhebungen von $\gamma_1$ bzw. $\gamma_2$ wie in obigem Lemma. Dann gibt es genau eine Homotopie
    \[ \tilde{H} : \I \times \I \to \tilde{X} \]
    von $\tilde{\gamma_1}$ und $\tilde{\gamma_2}$ relativ der Endpunkte.
  \end{lem}
\end{frame}

\iffalse
\begin{frame}
  \begin{definition}
    Ein \emph{lokaler Homöomorphismus} ist eine stetige Abbildung $f : Y \to Z$ zwischen topologischen Räumen, für die gilt:

    Für alle $y \in Y$ eine Umgebung $U_y \subseteq Y$ von $y$ existiert, sodass $f(U_y)$ offen und
    \[ f|_{U_y} : U_y \to f(U_y) \]
    ein Homöomorphismus ist.
  \end{definition}

  \begin{bem}<2->
    \begin{itemize}
      \item Jeder lokale Isomorphismus ist ein lokaler Homöomorphismus.
      \item Jede Überlagerungsabbildung $p : \tilde{X} \to X$ ist ein lokaler Homöomorphismus, aber nicht jeder lokale Homöomorphismus ist eine Überlagerungsabbildung.
    \end{itemize}
  \end{bem}
\end{frame}
\fi

\begin{frame}
  \begin{lem}[BH, I.3.28.]
    Sei $X$ ein wegzusammenhängender metrischer Raum, $\tilde{X}$ ein vollständiger metrischer Raum und $p : \tilde{X} \to X$ ein lokaler Homöomorphismus. Angenommen,
    \begin{itemize}
      \item $L(\tilde{\alpha}) \leq L(p \circ \tilde{\alpha})$ für alle Wege $\alpha : \I \to \tilde{X}$
      \item für alle $x \in X$ gibt es ein $r > 0$, sodass jedes $y \in B_r(x)$ mit $x$ durch eine eindeutige linear parametrisierte Geodäte $\gamma_y : \I \to B_r(x)$ verbunden ist und $\gamma_y$ stetig von $y$ abhängt.
    \end{itemize}
    Dann ist $p$ eine Überlagerung.
  \end{lem}

  \begin{folg}
    Sei $X$ ein Hadamard-Raum (vollständig, einfach zshgd, CAT($0$)) und $p \in X$. Dann ist $\exp_p : \tilde{X}_p \to X$ eine Überlagerung.
  \end{folg}
\end{frame}

\begin{frame}
  \begin{definition}
    Ein vollständiger, einfach zusammenhängender Längenraum mit Alexandrov-Krümmung $\leq 0$ heißt \emph{Hadamard-Raum}.
  \end{definition}

  \begin{lem}
    Sei $p : \tilde{X} \to X$ eine Überlagerungsabbildung, $\tilde{X} \not= \emptyset$, $X$ einfach zusammenhängend und $\tilde{X}$ wegzusammenhängend. Dann ist $p$ ein Homöomorphismus.
  \end{lem}

  \begin{folg}
    $exp_p : \tilde{X}_p \to X$ ist ein Homöomorphismus, wenn $X$ ein Hadamard-Raum ist.
  \end{folg}

  \begin{satz}[Cartan-Hadamard]<2->
    \begin{enumerate}
      \item Für alle Paare $p, q$ von Punkte in einem Hadamard-Raum gibt es genau eine verbindende Geodäte $\sigma_{pq}$.
      \item All diese Geodäten sind kürzeste Wege.
    \end{enumerate}
  \end{satz}
\end{frame}

% Induzierte Längenstrukturen. Notwendig?
% Lemma BH, II.4.7.
% Hilfslemma (1.108 bzw. BH, I.3.28)
% Letztendliche Formulierung des Satzes von Cartan-Hadamard?
% Eindeutigkeit von Geodäten
% "Alexandrov's patchwork": CAT(0)-Vergleichseigenschaft auch im Großen.
% Dazu: Alexandrovs Lemma, "Verklebelemma"

\framedgraphic{bilder/Picture15.jpg}

\begin{frame}
  \begin{satz}
    Sei $X$ ein Hadamard-Raum. Dann gilt für alle $x, y, z \in X$ mit verbindenden Kürzesten $\sigma_{xy}, \sigma_{yz}, \sigma_{xz}$: Die Winkelsumme des Dreiecks $\Delta (x, y, z)$ ist $\leq \pi$.
  \end{satz}

  \begin{bem}
    Dies ist äquivalent dazu, dass das Dreieck die Vergleichseigenschaft erfüllt. Somit sind in Hadamard-Räumen Dreiecke beliebiger Größe "`dünn"'.
  \end{bem}

  \begin{bew}
    "`Alexandrovs Flickwerk"'
  \end{bew}
\end{frame}

\framedgraphic{bilder/cat.jpg}

\end{document}