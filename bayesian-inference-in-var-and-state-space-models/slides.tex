\documentclass[10pt]{beamer}

\mode<presentation>
{
  \usetheme{amsterdam}
  %\usecolortheme{seahorse}
  \usefonttheme{default}
  %\useoutertheme[]{smoothbars}
  \setbeamertemplate{navigation symbols}{}
  \setbeamertemplate{caption}[numbered]
  \setbeamersize{text margin left=0.7cm}
  \setbeamersize{text margin right=0.7cm}
} 

\setlength{\leftmarginii}{8pt}
\setlength{\leftmarginiii}{8pt}

\usepackage[utf8x]{inputenc}
\usepackage[ngerman]{babel}

\usepackage{color}
\usepackage{mathtools}

\theoremstyle{definition}
\newtheorem*{bsp}{Beispiel}
\newtheorem*{prior}{Prior-Verteilung}
\newtheorem*{inference}{Inferenz}

\newcommand{\R}{\mathbb{R}} % Reelle Zahlen
\DeclareMathOperator{\var}{Var} % Varianz
%\DeclareMathOperator{\cov}{Cov} % Kovarianz
%\DeclareMathOperator{\cor}{Cor} % Korrelation
\newcommand{\Normal}{\mathcal{N}} % Gaußsche Normalverteilung
\newcommand{\Uniform}{\mathcal{U}} % Gaußsche Normalverteilung
\newcommand{\InverseWishart}{\mathcal{IW}} % Inverse Wishart Distribution
\DeclareMathOperator{\Vector}{vec} % Vektor (aus einer Matrix)

\newcommand{\new}{\mathrm{new}} % neuer Wert
\newcommand{\old}{\mathrm{old}} % alter Wert

\newcommand{\TODO}[1]{\textcolor{orange}{TODO: #1}}

\definecolor{StepOneColor}{rgb}{0.7,0.2,0.0}
\definecolor{StepTwoColor}{rgb}{0.2,0.7,0.0}
\definecolor{TypeInfoColor}{rgb}{0.4,0.4,0.4}

\newcommand{\stepOne}[1]{\textcolor{StepOneColor}{#1}}
\newcommand{\stepTwo}[1]{\textcolor{StepTwoColor}{#1}}
\newcommand{\typeInfo}[1]{\textcolor{TypeInfoColor}{#1}}

\DeclarePairedDelimiter\abs{\lvert}{\rvert} % Absolutwert

\makeatletter
\g@addto@macro\normalsize{%
  \setlength\abovedisplayskip{2pt}
  \setlength\belowdisplayskip{2pt}
  \setlength\abovedisplayshortskip{2pt}
  \setlength\belowdisplayshortskip{2pt}
}
\makeatother

\title{\TODO{Der Random-Walk Metropolis-Hastings-Algorithmus für Threshold-VAR-Modelle}}
\author{Tim Baumann}
\date{29. April 2016}

\begin{document}

\begin{frame}
  \titlepage
\end{frame}

\begin{frame}
  \tableofcontents
  % TODO: mehr wie im Buch
\end{frame}

% 3.4. "The random walk MH algorithm used in a Threshold VAR model"

\section[Threshold-VAR-Modell]{Der Random-Walk Metropolis-Hastings-Algorithmus für Threshold-VAR-Modelle}

\begin{frame}[t]
  \frametitle{Das Threshold-VAR-Modell}

  % TODO: Kovarianz von $v_t$ schreiben (anstand Varianz)
  \begin{align*}
    \text{(TVAR)} \enspace
    & \begin{cases}
      Y_t = \stepOne{c_1} + \sum_{j=1}^P \stepOne{\beta_1} Y_{t-j} + v_t, \enspace
      \var(v_t) = \stepOne{\Omega_1}
      & \text{wenn } S_t \leq \stepTwo{Y^*} \\[4pt]
      Y_t = \stepOne{c_2} + \sum_{j=1}^P \stepOne{\beta_2} Y_{t-j} + v_t, \enspace
      \var(v_t) = \stepOne{\Omega_2}
      & \text{wenn } S_t > \stepTwo{Y^*}
    \end{cases} \\
    & \text{wobei } S_t \coloneqq Y_{j, t-d} \enspace \text{(\emph{Threshold-Variable})} \\
    & \typeInfo{
      Y_t, v_t, c_1, c_2 \in \R^N, \enspace
      \beta_1, \beta_2 \in \R^{N \times N}, \enspace
      \Omega_1, \Omega_2 \in \R^{N \times N}, \enspace
      Y^* \in \R
    }
  \end{align*}

  Dabei wird die Threshold-Komponente~$j$ von~$Y$ und die Verzögerung~$d$ vom Anwender gewählt.

  \begin{bsp}<2->
    Makroökonomische Modellierung, wobei vermutet wird, dass die Stärke wirtschaftlicher Zusammenhänge (z.\,B. Multiplikator für Staatsausgaben) in Wirtschaftkrisen unterschiedlich groß ist wie in wirtschaftlich normalen oder guten Zeiten.
  \end{bsp}
\end{frame}

\begin{frame}
  \frametitle{Bayessche Inferenz im Threshold-VAR-Modell}
  % TODO: Kovarianz von $v_t$ schreiben (anstand Varianz)
  \begin{align*}
    \text{(TVAR)} \enspace
    & \begin{cases}
      Y_t = \stepOne{c_1} + \sum_{j=1}^P \stepOne{\beta_1} Y_{t-j} + v_t, \enspace
      \var(v_t) = \stepOne{\Omega_1}
      & \text{wenn } S_t \leq \stepTwo{Y^*} \\[4pt]
      Y_t = \stepOne{c_2} + \sum_{j=1}^P \stepOne{\beta_2} Y_{t-j} + v_t, \enspace
      \var(v_t) = \stepOne{\Omega_2}
      & \text{wenn } S_t > \stepTwo{Y^*}
    \end{cases} \\
    & \text{wobei } S_t \coloneqq Y_{j, t-d} \enspace \text{(\emph{Threshold-Variable})}
  \end{align*}

  \begin{prior}<2->
    \begin{itemize}
      \item<3-> Für den Threshold: \quad $p(\stepTwo{Y^*}) \sim \Normal(\overline{Y}^*, \sigma_{Y^*})$
      \item<4-> Für die \stepOne{VAR-Parameter}~$\stepOne{b_1}, \stepOne{b_2} \, \typeInfo{\in \R^{(1 + NP) \cdot N}}$ und $\stepOne{\Omega_1}, \stepOne{\Omega_2} \, \typeInfo{\in \R^{N \times N}}$ verwenden wir die Normal-Inverse-Wishart-Verteilung mit Dummy-Observations $X_{D,i} \, \typeInfo{\in \R^{k_i \times (1 + NP)}}$, $Y_{D,i} \, \typeInfo{\in \R^{k_i \times N}}$ ($i = 1,2$):
      \[
        \arraycolsep=1.5pt
        \begin{array}{rcl}
          p(b_i | \Omega_i) &\sim& \Normal(\Vector(B_{D,i}), \Omega_i \otimes (X_{D,i}^T X_{D,i})^{-1}), \\
          p(\Omega_i) &\sim& \InverseWishart(S_{D,i}, \TODO{T_{D,i} - ???})
        \end{array}
      \]
      \[
        \begin{array}{rrl}
          \text{wobei} \enspace
          & B_{D,i} & \coloneqq (X_{D,i}^T X_{D,i})^{-1} (X_{D,i} Y_{D,i}) \, \typeInfo{\in \R^{(1 + NP) \times N}} \\
          & S_{D,i} & \coloneqq (Y_{D,i} - X_{D,i} B_{D,i})^T (Y_{D,i} - X_{D,i} B_{D,i}) \, \typeInfo{\in \R^{N \times N}}
        \end{array}
      \]
    \end{itemize}
  \end{prior}
\end{frame}

\begin{frame}[t]
  \frametitle{Bayessche Inferenz im Threshold-VAR-Modell}
  \begin{enumerate}
    \item<2->[A.] Initialisierung: Wähle einen Startwert für den Treshold $\stepTwo{Y^*}$ \only<1-13>{\\ \small (z.\,B. den Durschnitt oder den Median der Werte $S_t$)}
    \item<3->[B.] Gibbs-Sampling: Wiederhole die Schritte
    \begin{enumerate}
      \item<4->[\stepOne{1.}] Sample die \stepOne{VAR-Parameter} gegeben dem Threshold~\stepTwo{$Y^*$}:
      \begin{itemize}
        \item
        \begin{onlyenv}<5-13>
          Beobachtung: Ist $Y^*$ bekannt, so zerfällt das Modell in zwei einfache VAR-Modelle, eines für das Regime $S_t \leq Y^*$, eines für $S_t > Y^*$.
        \end{onlyenv}
        \item<6-> Seien $Y_{1,t}$, $X_{1,t}$ die zum Regime $S_t \leq Y^*$ und $Y_{2,t}$, $X_{2,t}$ die zum Regime $S_t > Y^*$ zugehörigen Daten.
        \item
        \begin{onlyenv}<7-8>
        Ziehe \stepOne{$b_1, b_2, \Omega_1, \Omega_2$} aus der Posterior-Verteilung
        \[
          \begin{array}{rcl}
            p(b_i | \Omega_i, Y_{i,t}) &\sim& \Normal(\Vector(B_{i}^{*}), \Omega_i \otimes ((X_{i}^{*})^T X_{i}^{*})^{-1}), \\
            p(\Omega_i, Y_{i,t}) &\sim& \InverseWishart(S_{i}^{*}, \TODO{T_{i}^{*} - ???})
          \end{array}
        \]
        \[
          \arraycolsep=1.5pt
          \begin{array}{rrcl}
            \text{wobei} \enspace
            & B_{i}^{*} &\coloneqq& ((X_{i}^{*})^T X_{i}^{*})^{-1} (X_{i}^{*} Y_{i}^{*}) \\
            & S_{i}^{*} &\coloneqq& (Y_{i}^{*} - X_{i}^{*} B_{i}^{*})^T (Y_{i}^* - X_{i}^* B_{i}^*) \\
            & Y_i^{*} &\coloneqq& [Y_{i,t}, Y_{D,i}] \\
            & X_i^{*} &\coloneqq& [X_{i,t}, X_{D,i}]
          \end{array}
        \]
        \end{onlyenv}
        \item \begin{onlyenv}<9->
        Ziehe \stepOne{$b_1, b_2, \Omega_1, \Omega_2$} aus der Posterior-Verteilung $p(b_i | \Omega_i, Y_{i,t})$, $p(\Omega_i, Y_{i,t})$.
        \end{onlyenv}
      \end{itemize}
      \item<8->[\stepTwo{2.}] Führe einen Random-Walk-Metropolis-Hastings-Schritt für \stepTwo{$Y^*$} aus:
      \begin{itemize}
        \item<10-> Generiere eine Kandidaten $Y^{*}_{\new}$ durch einen Random-Walk-Schritt:
        \[
          Y^{*}_{\new} \coloneqq Y^{*}_{\old} + e, \quad
          e \sim \Normal(0, \sigma)
        \]
        \item<11-> Berechne die Akzeptanz-Wahrscheinlichkeit
        \begin{align*}
          \only<-16>{
            \alpha
            &= \frac{\pi(\phi^{G+1})}{\pi(\phi^{G})} \cdot \frac{q(\phi^G \,|\, \phi^{G+1})}{q(\phi^{G+1} \,|\, \phi^{G})}
            \visible<12->{
              = \frac{p(Y^{*}_{\new} \,|\, b_1, \Omega_1, b_2, \Omega_2, Y_t)}{p(Y^{*}_{\old} \,|\, b_1, \Omega_1, b_2, \Omega_2, Y_t)}
            } \\
          }
          \visible<13->{
            \visible<17->{\alpha} &= \frac{p(Y_t \,|\, b_1, \Omega_1, b_2, \Omega_2, Y^{*}_{\new}) \cdot p(Y^{*}_{\new})}{p(Y_t \,|\, b_1, \Omega_1, b_2, \Omega_2, Y^{*}_{\old}) \cdot p(Y^{*}_{\old})}
          }
        \end{align*}
        \vspace{-8pt}
        \begin{onlyenv}<15->
          \begin{align*}
            & p(Y_t \,|\, b_1, \Omega_1, b_2, \Omega_2, Y^{*}) = p(Y_{1,t} \,|\, b_1, \Omega_1, Y^{*}) \cdot p(Y_{2,t} \,|\, b_2, \Omega_2, Y^{*}) \\[-2pt]
            & \log p(Y_{i,t} \,|\, b_i, \Omega_i, Y^{*}) = \tfrac{T}{2} \log \abs{\Omega_i^{-1}} - \tfrac{1}{2} {\sum}_{t=1}^T (Y_{i,t} - X_{i,t} \tilde{b_i})^T \Omega_i^{-1} (Y_{i,t} - X_{i,t} \tilde{b_i})
          \end{align*}
        \end{onlyenv}
        \item<16-> Ziehe $u \sim \Uniform(0,1)$. Behalte $Y^{*}_{\new}$, falls $u < \alpha$, ansonsten verwerfe $Y^{*}_{\new}$.
      \end{itemize}
    \end{enumerate}
  \end{enumerate}
\end{frame}


\end{document}
